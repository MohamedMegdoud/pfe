
\documentclass[12pt,a4paper]{report}
\usepackage{fontspec}
\usepackage{polyglossia}
\setdefaultlanguage{arabic}
\setotherlanguage{english}
\newfontfamily\arabicfont[Script=Arabic]{Amiri}
\usepackage{geometry}
\geometry{a4paper, left=3cm, right=2.5cm, top=3cm, bottom=3cm}
\usepackage{fancyhdr}
\usepackage{titlesec}
\usepackage{setspace}
\usepackage{tocbibind}
\usepackage{graphicx}
\usepackage{hyperref}
\usepackage{lipsum}
\pagestyle{fancy}
\fancyhf{}
\fancyfoot[C]{\thepage}
\titleformat{\chapter}[display]{\normalfont\huge\bfseries}{\chaptername\ \thechapter}{20pt}{\Huge}
\titleformat{\section}{\Large\bfseries}{\thesection}{1em}{}
\renewcommand{\baselinestretch}{1.5}
\begin{document}

\begin{titlepage}
\centering
{\Large جـــــامعــة فرحات عباس سطيـــــــف 1- الجزائر-\\
كلية العلوم الاقتصادية والتجارية وعلوم التسيير\\
قســــم: العلوم الاقتصادية}\\[2cm]

{\Huge \textbf{مــــــذكـــــرة}}\\[0.5cm]

{\large مقـدمة ضمن متطلبات الحصول على شهادة المـاستر شعبة العلوم الاقتصادية\\
تخصــص: اقتصاد كمي}\\[2cm]

{\LARGE \textbf{أثر التحويلات الاجتماعية على ديناميكية معدلات التضخم في الجزائر خلال الفترة 2000-2022 باستعمال نماذج NARDL}}\\[2cm]

{\Large إعداد الطالب: \textbf{حنيوة صالح}}\\[0.5cm]
{\Large إشراف الأستاذ: \textbf{شلاعي فاتح}}\\[2cm]

{\Large تاريخ المناقشة: 29/06/2025}\\
{\Large السنة الجامعية: 2024-2025}
\end{titlepage}

\tableofcontents
\newpage

   جـــــامعــة فرحات عباس سطيـــــــف 1- الجزائر-

كلية العلوم الاقتصادية والتجارية وعلوم التسيير  
قســــم: العلوم الاقتصادية
          مــــــذكـــــرة
مقـدمة ضمن متطلبات الحصول على شهادة المـاستر شعبة العلوم الاقتصادية
تخصــص: اقتصاد كمي
الموضـــــــــــوع:
إعداد الطـلبة:      
 حنيوة صالح                                                                              
تــاريخ المنـــاقشة 29/06/2025
السنة الجامعية:2024-2025
جـــــامعــة فرحات عباس سطيـــــــف 1- الجزائر-

كلية العلوم الاقتصادية والتجارية وعلوم التسيير
قســــم: العلوم الاقتصادية
          مــــــذكـــــرة
مقـدمة ضمن متطلبات الحصول على شهادة المـاستر شعبة العلوم الاقتصادية
تخصــص: اقتصاد كمي
الموضـــــــــــوع:
إعداد الطـلبة:                                                                                              اشراف الاستاذ:











 حنيوة صالح                                                                                               شلاعي فاتح                                  
تــاريخ المنـــاقشة 29/06/2025
السنة الجامعية:2024-2025

الشكر: 
أحمد الله أولاً وأخيراً على توفيقه، ثم أتقدم بجزيل الشكر والعرفان إلى أستاذي المشرف شلاعي فاتح، الذي لم يبخل عليّ بتوجيهاته القيّمة وملاحظاته البنّاءة طيلة مراحل إعداد هذه المذكرة.
كما لا يفوتني أن أشكر أعضاء لجنة المناقشة على قبولهم مناقشة هذا العمل وعلى ملاحظاتهم التي ستكون موضع اعتبار.
وأخص بالشكر كل من ساهم في دعمي وتشجيعي خلال هذا المشوار العلمي، من أساتذة وزملاء وأفراد العائلة، لا سيما والديّ العزيزين، جزاهما الله عني كل خير..
الاهداء:
إلى كل من كانت له يد، ظاهرة أو خفية، في دعمي ومساندتي خلال هذا المشوار العلمي،
إلى من آمن بي وشجّعني، إلى من ساندني بكلمة أو فعل أو دعاء،
إلى كل من كان جزءاً من هذا الإنجاز،
أُهدي هذا العمل المتواضع… عربون شكر وامتنان.
تمهيد:
           تُعد التحويلات الاجتماعية إحدى الآليات الجوهرية التي تعتمدها السياسات العمومية لتحقيق التوازن الاجتماعي وتعزيز العدالة الاقتصادية، وذلك من خلال إعادة توزيع الدخل وتحسين مستويات المعيشة لدى الفئات الهشة وذوي الدخل المحدود. وتندرج هذه التحويلات، سواء كانت نقدية أو عينية، ضمن تدخلات الدولة لضمان الحماية الاجتماعية، خصوصاً في السياقات التي تتسم بعدم الاستقرار الاقتصادي أو التفاوتات الاجتماعية المتزايدة. 
            في المقابل، يشكل التضخم أحد أبرز التحديات التي تواجه الاقتصاد الكلي، لما له من تأثيرات مباشرة وغير مباشرة على القدرة الشرائية، توزيع الدخل، الادخار، والاستثمار. ويُنظر إلى التضخم على أنه ظاهرة متعددة الأسباب، تتداخل فيها العوامل النقدية والهيكلية والمؤسساتية، ما يجعل تحليل ديناميكيته يتطلب فهماً معمقاً لمجمل المتغيرات الاقتصادية والاجتماعية ذات الصلة.
            شهد الاقتصاد الجزائري منذ مطلع الألفية الثالثة تحولات هيكلية هامة على المستويين الاجتماعي والاقتصادي، كان من أبرز معالمها ارتفاع حجم التحويلات الاجتماعية في إطار سياسات الدولة الرامية إلى تحقيق العدالة الاجتماعية وتقليص الفوارق بين الفئات. فقد خصصت الحكومات المتعاقبة مبالغ ضخمة لدعم السلع الأساسية، الإعانات المباشرة وغير المباشرة، والتقاعد ومنح البطالة وغيرها من أشكال التحويلات التي تشكل جزءاً جوهرياً من النفقات العامة.
في المقابل، عرفت معدلات التضخم خلال نفس الفترة تذبذبات ملحوظة، تأثرت بعوامل داخلية وخارجية، من بينها تقلبات أسعار النفط، اختلالات هيكلية في الاقتصاد، وضغوط الطلب. وفي هذا السياق، تطرح إشكالية العلاقة بين التحويلات الاجتماعية ومعدل التضخم نفسها بقوة، خصوصاً في ظل الجدل القائم حول فعالية السياسات الاجتماعية من جهة، وانعكاساتها الاقتصادية على استقرار الأسعار من جهة أخرى. إن هذه العلاقة لا تبدو بالضرورة خطية أو متماثلة، بل قد تتخذ طابعاً غير متماثل، حيث قد يكون لارتفاع التحويلات الاجتماعية أثر مختلف عن انخفاضها، سواء في الأجل القصير أو الطويل. ومن هنا تبرز أهمية استعمال نماذج قياسية مرنة قادرة على التقاط هذا النوع من العلاقات، مثل نموذج الانحدار الذاتي للإبطاء الموزع غير الخطي (NARDL)، الذي يسمح بدراسة الأثر غير المتماثل للمتغيرات التفسيرية على المتغير التابع.


1- إشكالية الدراسة:
يمكن صياغة الإشكالية الرئيسة كما يلي: ما هو أثر التحويلات الاجتماعية على ديناميكية معدلات التضخم في الجزائر خلال الفترة 2000-2022؟، إضافة الى مجموعة من التساؤلات الفرعية التالية:
كيف تطورت التحويلات الاجتماعية ومعدلات التضخم في الجزائر خلال الفترة المدروسة؟
ما طبيعة العلاقة بين التحويلات الاجتماعية والتضخم؟
هل تؤثر التحويلات الاجتماعية بشكل متماثل على التضخم في حالتي الزيادة والنقصان؟
كيف يمكن نمذجة العلاقة بين التحويلات الاجتماعية والتضخم باستخدام نموذج NARDL؟
2- فرضيات الدراسة:
انطلاقاً من الإشكالية المطروحة، نقترح الفرضيات التالية:
الفرضية الرئيسية:
توجد علاقة غير متماثلة بين التحويلات الاجتماعية ومعدلات التضخم في الجزائر خلال الفترة 2000-2022.
الفرضيات الفرعية:
تشهد التحويلات الاجتماعية والتضخم في الجزائر تذبذبات مرتبطة بالتحولات الاقتصادية والسياسية.
تؤثر التحويلات الاجتماعية على التضخم بشكل مختلف في حالات الزيادة مقارنة بحالات النقصان.
يظهر أثر التحويلات الاجتماعية على التضخم بشكل أقوى في الأجل الطويل مقارنة بالأجل القصير
3-اهداف الدراسة: 
تسعى هذه الدراسة إلى تحقيق الأهداف التالية:
دراسة تطور التحويلات الاجتماعية ومعدلات التضخم في الجزائر خلال الفترة 2000-2022.
تحليل العلاقة بين التحويلات الاجتماعية والتضخم.
بناء نموذج قياسي باستخدام NARDL لقياس أثر التحويلات الاجتماعية على التضخم.
تحديد ما إذا كان هذا الأثر متماثلاً أو غير متماثل في الأجلين القصير والطويل.
4-أهمية الدراسة: 
تنبع أهمية هذه الدراسة من كونها تسلط الضوء على إحدى القضايا الحيوية في الاقتصاد الجزائري، والمتمثلة في العلاقة بين التحويلات الاجتماعية والتضخم، وهي علاقة تحمل أبعاداً اقتصادية واجتماعية متداخلة. ففي الوقت الذي تُعد فيه التحويلات الاجتماعية أداة رئيسية لتحقيق العدالة الاجتماعية وتعزيز الاستقرار الاجتماعي، فإنها قد تشكل في المقابل مصدر ضغط على الأسعار، خصوصاً إذا لم تكن موجهة بشكل فعّال. وتكتسي هذه الدراسة أهمية إضافية لاعتمادها على نموذج NARDL الذي يسمح بدراسة العلاقة غير المتماثلة بين المتغيرات، مما يُمكّن من فهم دقيق لتأثير الزيادة والنقصان في التحويلات الاجتماعية على معدلات التضخم في الأجلين القصير والطويل. كما أن تغطية الفترة الممتدة من 2000 إلى 2022، التي تميزت بتغيرات كبيرة في سياسات الدعم والتحويلات في الجزائر، يمنح الدراسة بعداً زمنياً مهماً، ويجعل نتائجها ذات قيمة تحليلية يمكن أن تفيد صانعي القرار في تحسين تصميم السياسات الاجتماعية والاقتصادية على حد سواء.
5- حدود الدراسة:
الحدود المكانية: تتمثل في أثر التحويلات الاجتماعية على ديناميكية معدلات التضخم في الجزائر ككل.
الحدود الزمانية: تم الاعتماد على الفترة 2000-2022.
6- صعوبات الدراسة:
تجدر الإشارة إلى أن هذه الدراسة لم تعتمد على توظيف المتغيرات الكمية الخاصة بمعدلات التضخم بشكل مباشر، سواء من خلال نماذج اقتصادية قياسية أو بيانات إحصائية مفصلة. ويعود ذلك إلى عدة اعتبارات، من أبرزها صعوبة الحصول على بيانات دقيقة ومستمرة حول مؤشرات التضخم في الجزائر، بالإضافة إلى تعقّد العوامل المتداخلة التي تؤثر في التضخم، مما يصعّب عزل أثر التحويلات الاجتماعية بشكل مستقل. وبناءً عليه، اعتمدت الدراسة في تحليلها على مقاربة نظرية تحليلية، تهدف إلى فهم العلاقة العامة بين التحويلات الاجتماعية والتضخم من منظور كيفي، مدعوم بالمعطيات المتوفرة والأدبيات الاقتصادية ذات الصلة.
كما واجهت الدراسة في جانبها التطبيقي صعوبات مرتبطة بندرة البيانات التفصيلية حول حجم وتوزيع التحويلات الاجتماعية، إضافة إلى ضعف انتظام النشر الإحصائي الرسمي، مما فرض حدودًا على إمكانيات التحقق الكمي من الفرضيات.
7-المنهجية المتبعة : 
تم الاعتماد في هذه الدراسة على المنهج الوصفي التحليلي لدراسة المفاهيم والنظريات المرتبطة بالتحويلات الاجتماعية والتضخم، والمنهج القياسي الكمي لقياس العلاقة بين المتغيرين محل الدراسة.
تم استخدام نموذج NARDL لقدرته على تحليل العلاقات غير المتماثلة بين المتغيرات، وذلك بالاعتماد على بيانات سنوية تغطي الفترة من 2000 إلى 2022، تم جمعها من مصادر رسمية مثل الديوان الوطني للإحصائيات (ONS)، وزارة المالية، والبنك العالمي.
1-مفاهيم عامة حول التحويلات الاجتماعية: 
1-1تعريف التحويلات الاجتماعية :
• هي سياسة تتدخل أكثر في اهتمام الدولة للحد من عدم المساواة، ومن أجل ذلك فإنها تنظم إعادة توزيع الثروة
• التحويلات الاجتماعية هي " النفقات التي يراد بها نقل القدرة الشرائية من الأغنياء الى الفقراء مثل الإعانات الاجتماعية والتأمينات الاجتماعية " .
•  "تُعدّ التحويلات الاجتماعية من أشكال نقل الموارد، سواء كانت نقدية أو عينية، إلى الأفراد أو الأسر الفقيرة أو الهشة، دون اشتراط مساهمات سابقة (أي دون مقابل اشتراكات)، حيث تُموَّل هذه التحويلات من الأموال العامة. وتتسم هذه التحويلات بكونها مباشرة، منتظمة، وقابلة للتنبؤ، وتهدف بالأساس إلى الحد من العجز الغذائي لدى الفئات المستفيدة، وحمايتهم من الصدمات المختلفة، لا سيما الاقتصادية والمناخية، وفي بعض الحالات، تُسهم كذلك في تعزيز قدراتهم الإنتاجية."
1- 2 أنواع التحويلات الاجتماعية: 
تتمثل أنواع التحويلات الاجتماعية فيما يلي:
1-2-1 دعم العائلات: ويشمل:
المنح ذات الطابع العائلي: تتضمن المنح العائلية، المنحة الدراسية، منحة الأجر الوحيد.
دعم التعليم: عن طريق تقديم الدعم لمراكز الخدمات الجامعية (منح الطلبة، الإطعام، النقل)، منح تلاميذ مؤسسات التعليم الأساسي والثانوي، والإطعام المدرسي، المنح الدراسية بالخارج، منحة التمدرس للتلاميذ المعوزين.
دعم أسعار المواد الاستهلاكية الأساسية: والمتمثلة في الحليب، الحبوب، السكر، الزيت الغذائي.
دعم الكهرباء والغاز والماء: تتضمن كهرباء الريف، التوزيع العمومي للغاز، التعويض عن تخفيضات فواتير الكهرباء في ولايات الجنوب.
1-2-2 الدعم الصحي: يتضمن دعم المؤسسات الاستشفائية، الصحة المدرسية، النفقات المتعلقة بالخدمات المقدمة في إطار اتفاقية التعاون الطبي.
1-2-3 دعم السكن: خصصت الجزائر موارد مالية ضخمة لهذا القطاع، بهدف تخفيف حدة أزمة السكن وتحسين المستوى لمعيشي للأفراد من خلال أنواع الدعم المقدم في هذا المجال، إما عبر إعانات مباشرة على شكل مساعدات مالية، بهدف تهيئة قطع أرضية أو حيازة ملكية خاصة بالمساكن أو توسيعها، وإما بإعانات مالية غير مباشرة تكون من خلال دعم برامج السكنات عن طريق تخفيض قيمة الإيجار أو تخفيض الفيمة السوقية للسكن وفقا لشروط يحددها التنظيم.
1-2-4 دعم معاشات المتقاعدين: تعتبر منظومة التقاعد إحدى آليات الضمان الاجتماعي التي تهدف إلى التغلب على فقدان الدخل بسبب الشيخوخة أو الوفاة أو العجز لصالح الأفراد، وبغرض حماية قدرتهم الشرائية، قامت الجزائر سنة 2006 بإنشاء الصندوق الوطني لاحتياطات التقاعد هدفه دعم وتكوين احتياطات مالية للمساهمة في استمرارية وديمومة نظام التقاعد، كما توجد إجراءات أخرى تكون على عاتق الدولة تسمح بالاستفادة من التقاعد فائدة بعض الأشخاص كالمجاهدين.
1-2-5 دعم المجاهدين: تتمتع الجزائر بنظام فريد في مجال التكفل بالمجاهدين وذوي الحقوق، من خلال إعطاء أهمية في تحسين الوضعية الاجتماعية، والصحية والنفسية لفائدتهم، إضافة إلى الحفاظ على الذاكرة الوطنية، ويتجلى ذلك من خلال:
دعم التغطية الاجتماعية للمجاهدين وذوي الحقوق في مواصلة عمليات التكفل الطبي وتحسين الخدمات والامتيازات المقدمة لهم.
اقتناء تجهيزات وإنجاز مراكز الراحة لفائدة المجاهدين ومعطوبي الحرب، التكفل بنفقات كافة خدمات النقل.
3-1الآثار المترتبة عن سياسة التحويلات الاجتماعية:
 يترتب عن سياسة التحويلات الاجتماعية العديد من الاثار أهمها: 
ارتفاع معدلات التضخم وزيادة أسعار الصرف بفعل العجز الحاصل في الموازنة العامة والاقتراض بنوعيه الداخلي والخارجي.
الزيادة في التبذير والإسراف وسوء استخدام السلع والخدمات المدعمة نتيجة لانخفاض أسعارها.
ان الهدف من التحويلات الاجتماعية هو تحقيق العدالة الاجتماعية ، غير ان السلع و الخدمات المدعمة يستفيد منها الفقراء و الأغنياء و هو ما يتنافى مع هذا المبدأ.

2-مفاهيم حول التضخم :
يمتلك التضخم العديد من التعريفات نذكر منها مايلي 
2-1 تعريف التضخم:
        المقصود بالتضخم خلال الفترة السابقة للحرب العالمية الأولى، يختلف عن لتعريف الذي ساد بعدها في الفترة ما بين الحربين العالميتين، فكان التضخم لدى الكثير من الاقتصادين: هو إصدار النقود بصفة مطلقة، دون النظر إلى وجود عوامل أخرى، ولكن هذا المفهوم قد تغير فيما بعد، حيث أصبح المقصود به هو فائض النقد على فائض السلع والخدمات، ولا شك أن الأخذين بهذا المعنى قد تأثروا بالنظريات والمفاهيم الكينزية التي سادت بين الحربين، فاختلف مفهوم التضخم خلال نفس الفترة الزمنية باختلاف وجهات نظر المفكرين. 
            المفهوم البسيط للتضخم هو زيادة كمية النقود بدرجة تنخفض معها قيمة النقود أو على أنه الارتفاع
المستمر في المستوى العام للأسعار في دولة ما والناجم عن فائض الطلب عما هو معروض من السلع والخدمات خلال فترة زمنية معينة.
2-أنواع التضخم:
نوجز فيما يلي اهم أنواع التضخم: 
1-2-2 التضخم الصحيح او الأصيل : 
وهو التضخم الذي يتحقق حين لا يقابل الزيادة في الطلب الكلي زيادة معادلة في الإنتاج ولذلك فإن أثرذلك ينعكس في ارتفاع المستوى العام للأسعار على أن ذلك لا يمنع ارتفاع الأسعار حتى قبل الوصول إلى حالة الاستخدام الشامل.
2-2-2 التضخم المتدرج او الزاحف : 
ويحدث هذا النوع من الضخم عندما تزداد القوة الشرائية بصفة دائمة بنسبة أكبر من نسبة الزيادة في عرض كل من السلع وعوامل الإنتاج وخاصة خدمات العمل.
  التضخم المكبوت: 3-2-2
وهو التضخم الذي يمثل حالة تمنع فيها الأسعار من الارتفاع عن طريق سياسات تمثل بوضع ضوابط وقيود تحد من الانفاق الكلي وتحول دون ارتفاع الأسعار على ان ذلك لا يمنع الجمهور من تجميع موجودات نقدية سائلة كبيرة يمكن تحويلها الى قوة شرائية فعالة في وقت لاحق.
4-2-2 التضخم المفرط : 
هو التضخم الذي ترتفع فيه الأسعار بمعدلات عالية جدا وتزداد فيه سرعة تداول النقود، وتتوقف فيه النقود على العمل كمستودع للقيم. فإذا استمع ذلك الوضع فإنه يؤدي الى انهيار النظام النقدي وتنهار معه قيمة الوحدة النقدية. 
3-2 محددات التضخم :
تتمثل محددات التضخم في مايلي :
2-3-1 التضخم العائد لحجم الطلب او ما يعرف بتضخم سحب الطلب:
يُعد هذا النوع من التضخم نتيجة مباشرة لارتفاع الطلب الكلي على السلع والخدمات دون أن يقابله نمو مماثل في العرض، الأمر الذي يؤدي إلى ارتفاع تدريجي في المستوى العام للأسعار. وتظهر هذه الحالة بوضوح في ظروف التشغيل الكامل للموارد الاقتصادية، حيث لا تكون هناك طاقات إنتاجية إضافية يمكن استغلالها لتلبية الطلب المتزايد، مما يدفع الأسعار نحو الارتفاع. كما أن سعي الأفراد للحفاظ على قدرتهم الشرائية في ظل ارتفاع الأسعار، من خلال زيادة إنفاقهم أو مطالبتهم برفع الأجور، يفاقم من حدة التضخم ويؤدي إلى حلقة تضخمية متواصلة.
ويمكن الحد من هذا النوع من التضخم من خلال تبني سياسات مالية ونقدية انكماشية تهدف إلى تقليص الطلب الكلي، مثل زيادة الضرائب للحد من الإنفاق، وطرح سندات حكومية بأسعار فائدة مغرية لجذب السيولة من الأفراد والمؤسسات، إضافة إلى فرض سقوف ائتمانية على البنوك التجارية، مما يحد من قدرتها على خلق النقود وضخها في الاقتصاد.
2-3-2 التضخم الناشئ هن التكاليف:
ينشأ هذا التضخم نتيجة لمحاولة بعض المنتجين أو النقابات العمال أو كليهما رفع أسعار منتجاتهم وتعويضهم عن مستويات تفوق تلك المستويات التي يمكن أن تتحقق في حالات المنافسة الكاملة، ونظراً لأن الأسعار والأجور هي دخول بقدر ما هي تكاليف، فإن نشوء مثل هذا الأمر يصبح ممكناً.
فالمنتجون الذين يجنون من معدلات ربح أعلى، والعمال الذين يرغبون في أجور أكبر كلٌّاهما يتسبب في ارتفاع تكاليف الإنتاج، وبالتالي ارتفاع المستوى العام للأسعار، وينتج عن هذا التضخم الناشئ عن ارتفاع التكاليف انخفاض في حجم الكلي للسلع والخدمات. ويمكن معالجة هذا النوع من التضخم عن طريق ربط ارتفاع الأجور بالإنتاجية أو بالقدرة الإنتاجية بحيث لا يتم زيادة الأسعار بمجرد ارتفاع الأجور إلا بقدر الزيادة المتوقعة في الأسعار
3-3-2 التضخم المشترك : 
ينشأ هذا النوع من التضخم نتيجة السببين الأولين سواء أى تضخم الطلب وتضخم التكلفة، بمعنى زيادة في حجم النقود المتداولة بين أيدي الأفراد والمؤسسات، بدون نمو في حجم الإنتاج أو ثبات في الإنتاج وفي نفس الوقت زيادة وتكاليف بعض عناصر الإنتاج، مما يؤدي إلى ارتفاع الأسعار بشكل عام، وخاصة أسعار الفائدة على القروض الاستهلاكية أو أسعار المواد الخام الأولية، مما يؤدي إلى ارتفاع المستوى العام للأسعار للسلع والخدمات.
وبذلك فإن السبب المباشر لهذا التضخم هو التضخم المشترك بحسب ما ذكرنا سابقاً، والذي ناتج عن سياسات نقدية حدد من حجم النقود المتداولة، بالإضافة إلى القدرة الإنتاجية أو الإنتاج في حد ذاته . 
4-3-2 التضخم المستورد : 
ويعرف هذا التضخم المستورد على أنه الزيادة المتسارعة والمستمرة في أسعار السلع والخدمات النهائية كالملابس الجاهزة، والأطعمة الجاهزة والأجنبية المستوردة من الخارج. أي تصدير الدول وخاصة النامية هذا التضخم كما هو موجود في العالم الخارجي.
3- الدراسات السابقة:
  • دراسة جلول صفية وبو قناديل محمد سنة 2022 : تحت عنوان " التحويلات الاجتماعية و اثرها على عجز الميزانية العامة في الجزائر: دراسة قياسية باستخدام نموذج ARDL خلال الفترة 1990 -2020 " هدفت إحدى الدراسات إلى تحليل أثر التحويلات الاجتماعية على عجز الميزانية العامة في الجزائر خلال الفترة الممتدة من 1990 إلى 2020، وذلك باستخدام نموذج الانحدار الذاتي للإبطاء المُوزَّع (ARDL). اعتمدت الدراسة على المنهج التحليلي الكمي، حيث تم استخدام البيانات والإحصاءات الرسمية لتحليل العلاقة بين التحويلات الاجتماعية وعجز الميزانية. وبيّنت الدراسة أهمية التحويلات الاجتماعية كأداة من أدوات السياسة الاجتماعية التي تنتهجها الدولة الجزائرية، خاصة في ظل التوسع الكبير في الإنفاق العمومي الاجتماعي المصنّف ضمن نفقات التسيير، وقد ركّزت الدراسة على الإطار النظري لعجز الميزانية العامة إلى جانب تحليل واقع التحويلات الاجتماعية في الجزائر. وتوصلت نتائج التحليل إلى وجود أثر معنوي طويل الأجل للتحويلات الاجتماعية على عجز الميزانية، حيث تبين أن زيادة هذه التحويلات بنسبة 1% تؤدي إلى ارتفاع عجز الميزانية بنسبة تقديرية تبلغ 1.429%. كما أظهرت النتائج وجود أثر قصير الأجل ذو دلالة إحصائية عند مستوى معنوية (Prb ≤ 0.05)، مشيرة إلى أن طبيعة تأثير التحويلات الاجتماعية على الميزانية قد تكون موجبة أو سالبة حسب مخصصاتها، والتي تتأثر بدورها بتقلبات أسعار النفط باعتباره المصدر الأساسي للإيرادات العمومية في الجزائر.
• دراسة سي محمد كمال وبن هدي إكرام: تحت عنوان "تأثير الدعم الحكومي على الاستهلاك، الموازنة والنمو الاقتصادي في الجزائر" تهدف هذه الدراسة إلى تحليل أثر سياسة الدعم الحكومي على كل من عجز الموازنة، والاستهلاك، والنمو الاقتصادي في الجزائر. وقد أظهرت النتائج أن هذه السياسة لم تحقق الأهداف المرجوة منها، حيث لم تسهم في تحفيز النمو الاقتصادي بشكل فعّال، كما انعكست سلبًا على التوازنات المالية للدولة. فقد أدت إلى تفاقم عجز الموازنة، فضلاً عن تأثيرها في رفع مستويات الأسعار، مما ساهم في تشوهات اقتصادية زادت من الأعباء المالية على الحكومة، وأثرت على فعالية السياسات الاقتصادية الكلية. 
• دراسة حسينة جواني وسليم العمراوي سنة 2024: تحت عنوان "دراسة قياسية لأثر التحويلات الاجتماعية على مؤشرات الاستقرار الاقتصادي في الجزائر باستعمال نموذج ال VAR للفترة (1993-2020) تهدف هذه الدراسة إلى إبراز مدى تأثير التحويلات الاجتماعية في تحقيق الأهداف الاقتصادية، من خلال تحليل سلوك متغيرات مربع كالدور في الجزائر خلال الفترة الممتدة من 1993 إلى 2020. ولتحقيق هذا الهدف، تم اعتماد المنهج الوصفي لتحليل واقع المتغيرات الاقتصادية المدروسة، بالإضافة إلى استخدام نموذج الانحدار الذاتي المتجه (VAR) لدراسة العلاقة الديناميكية بين التحويلات الاجتماعية ومكونات مربع كالدور.
وقد أظهرت نتائج الدراسة أن الأداء الاقتصادي في الجزائر اتسم بالتذبذب خلال فترة الدراسة، حيث لوحظ تحسن في بعض محاور مربع كالدور يقابله تدهور في محاور أخرى. كما كشفت النتائج عن وجود علاقة عكسية بين نسبة التحويلات الاجتماعية وكل من معدل النمو الاقتصادي ورصيد ميزان المدفوعات، في حين تم تسجيل علاقة طردية مع كل من معدلات البطالة والتضخم. وتدل هذه النتائج على أن سياسة التحويلات الاجتماعية، رغم طابعها التضامني والاجتماعي، كان لها تأثير نسبي سلبي على تحقيق الأهداف الاقتصادية الكبرى، مما يبرز محدودية فعاليتها في دعم الاستقرار والنمو الاقتصادي.
.• دراسة قدار مريم وعيدودي فاطمة الزهرة سنة 2019 : تحت عنوان "دراسة تحليلية للتحويلات الاجتماعية في الجزائر خلال الفترة (2000-2018) " : تُعد التحويلات الاجتماعية في الجزائر أحد المكونات الأساسية لسياسة الدعم الحكومي، حيث تهدف الدولة من خلالها إلى ضمان مستوى معيشي كريم وتعزيز مبدأ العدالة الاجتماعية. وتهدف هذه الدراسة إلى تحليل سياسة التحويلات الاجتماعية التي تُعد على الرغم من أهميتها الاجتماعية والاقتصادية، عبئًا متزايدًا على الإنفاق العام، مما يستدعي إعادة النظر في آلياتها وتوجيهها نحو الفئات الأكثر استحقاقًا بما يضمن فعاليتها واستدامتها. 
وقد توصلت الدراسة إلى عدة نتائج، أبرزها أن الحكومة الجزائرية أولت اهتمامًا خاصًا للقطاعات الاجتماعية، لا سيما منذ بداية الألفية الثالثة، من خلال تخصيص موارد معتبرة لهذه التحويلات. كما بيّن التحليل أنه على الرغم من الانخفاض التدريجي في قيمة هذه التحويلات خلال السنوات الأخيرة، إلا أن الدولة حافظت على نفس التوجه الاجتماعي، ما يعكس التزامها بالبعد الاجتماعي في سياستها الاقتصادية، ولو على حساب بعض التوازنات المالية.
• دراسة سندس عبدربه، روميسة بن عدي مروة بن عمر سنة 2023: تحت عنوان "الآثار غير المتماثلة لأسعار النفط على التضخم في الجزائر، دراسة قياسية خلال الفترة (1985-2022) " : تهدف هذه الدراسة إلى تحليل الأثر غير المتماثل لتقلبات أسعار النفط على معدل التضخم في الجزائر خلال الفترة الممتدة من 1985 إلى 2022، وذلك بالاعتماد على منهجية الانحدار الذاتي للفجوات الزمنية المبطأة غير الخطية (NARDL). ولتحقيق أهداف الدراسة، تم بناء نموذج اقتصادي قياسي يضم متغيرين رئيسيين: أسعار النفط كمتغير مستقل، ومعدل التضخم كمتغير تابع.
وقد أظهرت نتائج التقدير وجود علاقة عكسية بين أسعار النفط ومعدل التضخم سواء على المدى القصير أو الطويل، حيث تبين أن ارتفاع أسعار النفط يؤدي إلى تراجع معدل التضخم، في حين أن انخفاض أسعار النفط يسهم في ارتفاعه. 
• دراسة حسين بن العارية عبد القادر عبد الرحمان سنة 2018 : تحت عنوان "تحليل ديناميكية التضخم في الجزائر خلال الفترة (1980-2014)" : تهدف هذه الدراسة إلى تحليل ديناميكية التضخم في الجزائر خلال الفترة الممتدة من 1980 إلى 2014، بهدف تحديد أبرز العوامل المفسرة لظاهرة التضخم، وتقييم مدى قدرة الاقتصاد الوطني على امتصاص الصدمات التضخمية واستعادة توازنه. ولتحقيق هذا الهدف، تم تقدير العلاقة طويلة الأجل بين معدل التضخم وأهم المتغيرات الاقتصادية المؤثرة فيه، مع اختبار وجود علاقة تكامل بين هذه المتغيرات باستخدام منهجية إنجل–غرانجر ذات الخطوتين. كما تم استخدام نموذج تصحيح الخطأ (ECM) لتحليل سلوك التضخم على المدى القصير، بالإضافة إلى اختبار السببية لتحديد اتجاه العلاقة بين المتغيرات المدروسة.
وقد توصلت نتائج الدراسة إلى ما يلي:
يُعد كل من الإنفاق الحكومي، معدل نمو المعروض النقدي، التضخم العالمي، وأسعار النفط من أهم مصادر التضخم في الجزائر على المدى الطويل.
يُعتبر الإنفاق الحكومي ومعدل نمو المعروض النقدي من أبرز مسببات التضخم في الجزائر على المدى القصير.
يتمتع الاقتصاد الجزائري بقدرة كبيرة على تصحيح الاختلالات والعودة إلى وضع التوازن في حال حدوث أي صدمة تضخمية.
• دراسة عزي خليفة، مسعودي زكرياء، شليق عبد الجليل سنة 2022: تحت عنوان "تحليل مشكلة التضخم في الجزائر باستخدام نموذجيFMOLS وECM، دراسة قياسية للفترة (1980-2017) ": تهدف هذه الدراسة إلى تحليل تطور ظاهرة التضخم في الجزائر خلال الفترة 1980–2017، من خلال تحديد أبرز المتغيرات الاقتصادية المؤثرة على سلوك التضخم. ولتحقيق هذا الهدف، تم استخدام نموذج المربعات الصغرى المصححة كلياً (FMOLS) لتقدير العلاقة طويلة الأجل بين معدل التضخم والمتغيرات الاقتصادية، ونموذج تصحيح الخطأ (ECM) لتحليل السلوك القصير الأجل للتضخم.
 أظهرت النتائج أن النمو الاقتصادي، سعر الفائدة، سعر الصرف، وقيمة الواردات تُعد من المحددات الرئيسية لسلوك التضخم على المدى الطويل، بينما تُعتبر نفقات الأجور العامل الأكثر تأثيراً على التضخم في الأجل 
القصير.

• دراسة قليل لقمان وبو قناديل محمد سنة 2024: تحت عنوان "دراسة تحليلية لأثر تقلبات أسعار النفط على التحويلات الاجتماعية في الجزائر خلال الفترة (2022 -2000) تهدف هذه الدراسة إلى تحليل تأثير تقلبات أسعار النفط على مستوى التحويلات الاجتماعية في الجزائر خلال الفترة الممتدة من 2000 إلى 2022، وذلك بالاعتماد على المنهج الوصفي والتحليلي للبيانات الإحصائية المتاحة. وقد أظهرت النتائج التي تم التوصل إليها أن التحويلات الاجتماعية تشهد نمواً متواصلاً، كما يتبين أن هذه التحويلات تتأثر بارتفاع أسعار النفط على المدى الطويل، بينما لا يظهر لها تأثير ملحوظ في الأجل القصير. حيث أن أي ارتفاع في أسعار النفط يقابله غالباً ارتفاع في حجم التحويلات الاجتماعية، في حين أن انخفاض الأسعار لا يؤدي إلى تراجع مباشر في هذه التحويلات على المدى القصير.
• دراسة كافية عيدوني سنة 2021: تحت عنوان "اثر الدعم الاجتماعي على الميزانية العامة للدولة في الجزائر – دراسة قياسية للفترة (1993 -2018) باستخدام نموذج - ARDL - " تناولت هذه الدراسة موضوع سياسة الدعم الاجتماعي في الجزائر وأثرها على الميزانية العامة للدولة، حيث سعت إلى تسليط الضوء على تطور أشكال هذه السياسة، لاسيما التحويلات الاجتماعية، من خلال تحليل مكوناتها خلال الفترة الممتدة من 1993 إلى 2018. وقد تم اعتماد نموذج الانحدار الذاتي للفجوات الزمنية الموزعة المبطئة (ARDL) كأداة تحليلية، إلى جانب استخدام المنهج التحليلي الكمي القائم على جمع البيانات الإحصائية وتحليلها بشكل منطقي بهدف الوصول إلى نتائج دقيقة. وقد خلصت الدراسة إلى أن كل زيادة في الدعم الاجتماعي والتحويلات الاجتماعية تؤدي إلى تفاقم عجز الميزانية العامة للدولة. وعلى الرغم من أن التحويلات الاجتماعية تشكل عنصراً ديناميكياً ضمن بنود الميزانية وتحظى بخصوصية من حيث توزيعها على عدة أبواب، إلا أنها تمثل أحد العوامل الرئيسة المساهمة في العجز المالي الذي تعرفه الميزانية العامة في الجزائر . 
•  دراسة صلاح الدين سويسي سنة 2022 : تحت عنوان "أثر الانفاق الجكرمي على معدل التضخم في الدول المغاربية دراسة مقارنة :الجزائر، تونس، المغرب، خلال الفترة ( 1990 -2018) " : تسعى هذه الدراسة إلى استقصاء تأثير الإنفاق الحكومي على معدلات التضخم، من خلال تحليل مقارن يشمل ثلاث دول من منطقة المغرب العربي، وهي: الجزائر، تونس، والمغرب، وذلك خلال الفترة من 1990 إلى 2018. وقد تم اعتماد مجموعة من المتغيرات الاقتصادية شملت كلًا من معدل التضخم، حجم الإنفاق الحكومي، العرض النقدي، بالإضافة إلى معدل النمو الاقتصادي. ولتقدير العلاقة بين هذه المتغيرات، تم توظيف نموذج الانحدار الذاتي للفجوات الزمنية المتباطئة (ARDL). وقد كشفت نتائج الدراسة عن وجود علاقة توازنية طويلة الأجل بين الإنفاق الحكومي والتضخم في الدول الثلاث، إذ أظهرت أن ارتفاع الإنفاق الحكومي بنسبة 1% يؤدي إلى انخفاض في معدل التضخم يُقدّر بـ 0.93% في الجزائر، و0.56% في تونس، و0.50% في المغرب. 
• دراسة جواهرة صليحة و ششوي حسنى سنة 2021 : تحت عنوان " محددات التضخم في الجزائر خلال الفترة (1980 -2018) دراسة قياسية باستخدام نماذج ARDL " : تهدف هذه الدراسة إلى تحديد أبرز المتغيرات الاقتصادية التي تسهم في توجيه المسار التضخمي في الاقتصاد الجزائري خلال الفترة الممتدة من 1980 إلى 2018. ولتحقيق هذا الهدف، تم أولًا استعراض أبرز النظريات الاقتصادية المفسرة لظاهرة التضخم، بالإضافة إلى عدد من الدراسات التجريبية السابقة المتعلقة بمحدداته. بعد ذلك، تم إجراء تحليل قياسي لمعادلة محددات التضخم باستخدام نموذج الانحدار الذاتي للفجوات الزمنية المتباطئة (ARDL). وقد أظهر اختبار الحدود (Bounds Test) وجود علاقة توازنية طويلة الأجل بين معدل التضخم ومحدداته. كما خلصت النتائج إلى أن كلًا من الناتج الداخلي الخام، الواردات، وسعر الصرف تُعد من أبرز العوامل المؤثرة في التضخم على المديين القصير والطويل ضمن السياق الاقتصادي الجزائري خلال فترة الدراسة.
• دراسة  فطوم حوحو، حمزة طيبي و سهام عيساوي سنة 2023 : تحت عنوان " مستقبل التحويلات الاجتماعية في الجزائر في ظل تفاقم عجز الموازنة العامة خلال الفترة (2009-2018) – محاولة إصلاح سياسة الدعم الحكومي " : تسعى هذه الدراسة إلى استجلاء أثر عجز الموازنة العامة على سياسة التحويلات الاجتماعية في الجزائر خلال الفترة 2009–2018، وهي مرحلة عرفت تقلبات حادة في أسعار النفط على المستوى العالمي، ما ألقى بظلاله على التوازنات المالية للدولة. وعلى الرغم من الضغوط الاقتصادية المتزايدة وتفاقم عجز الموازنة، أظهرت نتائج الدراسة تمسك الدولة بسياسة التحويلات الاجتماعية، حرصًا منها على حماية الفئات الهشة ومحدودي الدخل، وتجسيدًا لالتزامها بالبعد الاجتماعي كركيزة أساسية في سياستها الاقتصادية، حتى في أحلك الظروف.
4- فجوة البحثية:
رغم تعدد الدراسات التي تناولت التحويلات الاجتماعية في الجزائر، إلا أن معظمها ركّز على علاقتها بعجز الميزانية العامة أو بمؤشرات أخرى كالنمو الاقتصادي والعدالة الاجتماعية، كما هو الحال في دراسات جلول صفية (2022)، كافية عيدوني (2021)، وجواني حسينة (2024). في المقابل، لم تحظَ العلاقة المباشرة بين التحويلات الاجتماعية ومعدلات التضخم بالدراسة الكافية، سواء من حيث قياس مدى تأثير هذه التحويلات على السلوك السعري العام، أو من حيث تحليل ديناميكية هذه العلاقة عبر الزمن، خاصة في ظل الأزمات الاقتصادية المتعاقبة التي عرفتها الجزائر خلال الفترة الممتدة من 2000 إلى 2022. كما أن الدراسات التي تناولت التضخم ركّزت بالأساس على محددات تقليدية مثل أسعار النفط، الإنفاق الحكومي، العرض النقدي، وسعر الصرف، دون دمج التحويلات الاجتماعية كمتغير مستقل مؤثر في تشكيل الضغوط التضخمية. وعليه، تسعى هذه الدراسة إلى سدّ هذه الفجوة من خلال تحليل قياسي لأثر التحويلات الاجتماعية على ديناميكية معدلات التضخم في الجزائر، مع الأخذ بعين الاعتبار السياق الاقتصادي المتغير خلال فترة الدراسة.
تمهيد:
عرض فيما يلي منهج الدراسة المستخدم في الجانب النظري والقياسي، مصادر جمع البيانات المستخدمة في الدراسة القياسية، بالإضافة إلى عرض أهم الخطوات للنموذج الاحصائي، والأساليب والأدوات المستخدمة في التحليل والتقدير بما يحقق أهداف الدراسة.
1- منهج الدراسة:
يُستخدم نموذج ARDL لتحليل العلاقة بين المتغيرات على المدى القصير والطويل، خاصة عندما تكون السلاسل الزمنية مزيجًا من I (0) وI (1). غير أن هذا النموذج يفترض وجود علاقة خطية بين المتغيرات، وهو ما قد لا يعكس الواقع الاقتصادي بدقة، خصوصًا إذا كانت التأثيرات غير متماثلة. لذلك، تم اعتماد نموذج NARDL كامتداد لـ ARDL، كونه يسمح بدراسة الأثر غير المتوازن للتغيرات الإيجابية والسلبية في التحويلات الاجتماعية على معدل التضخم، مما يوفر فهماً أكثر دقة وديناميكية للعلاقة بين المتغيرين.
2-مصادر جمع البيانات:
تم الاعتماد على العديد من المصادر كالديوان الوطني للإحصاء، وزارة المالية الوطنية وعرض مشروع قانون المالية –DGPP.
3- المجتمع و العينة:
يتمثل مجتمع الدراسة في كافة البيانات الاقتصادية المتعلقة بالاقتصاد الجزائري خلال الفترة الممتدة من 2000 إلى 2022، وبشكل خاص البيانات المرتبطة بالتحويلات الاجتماعية ومعدل التضخم، باعتبارهما المتغيرين الأساسيين في هذه الدراسة. وقد تم اختيار هذه الفترة لما تشهده من تحولات اقتصادية بارزة تؤثر في طبيعة العلاقة بين الإنفاق الاجتماعي والتضخم. أما عينة الدراسة، فهي عبارة عن سلسلة زمنية سنوية تضم القيم المسجلة للتحويلات الاجتماعية ومعدل التضخم. وتم الحصول على هذه البيانات من مصادر رسمية مثل الديوان الوطني للإحصائيات ووزارة المالية الوطنية، بهدف ضمان دقة التحليل القياسي وموثوقية النتائج المستخلصة.
4-أدوات الدراسة 
4-1 تعريف نموذج ARDL:
        يعد نموذج الانحدار الذاتي للفجوات الزمنية الموزعة المتباطئةARDL دمجا لكل من نماذج الانحدار الذاتي ونماذج فترات الإبطاء الموزعة، والذي يعتبر من بين المنهجيات الحديثة نسبيا، كما يسمح هذا النموذج باختبار العلاقة السببية بين متغيرين أو أكثر (متغير تابع من جهة ومتغير أو متغيرات مستقلة من جهة أخرى). ولقد شاع بين الباحثين استخدام منهجية ARDL بسبب بعض الميزات التي توفرها هذه المنهجية، فمن جهة هي تسمح بفحص العلاقة السببية بين المتغيرات في الأجلين الطويل والقصير، ومن جهة أخرى فهي تأخذ عدد كافي من فترات الإبطاء مما يساهم في الحصول على نتائج أفضل وأكثر دقة.
4-2 الصيغة العامة لنموذج ARDL:
     Y= α-  Β1X1+ β2X2- β3X3- β4X4 + β5X5 +.... +ui  
بحيث :
Y: المتغير التابع
X: المتغير المستقل 
α: الحد الثابت للدالة
Β1، β2، β3، β4، β5 : معاملات الانحدار 
4-3 نموذج NARDL 
      یعتبر أسلوب NARDL للتكامل المشترك  توسيعا أو تعميما للتقدير الخطي لأسلوب الانحدار الذاتي ذو الفجوات الزمنیة المبطئة للتكامل المشترك ARDL	. بحیث یأخذ بعین الاعتبار احتمالية اللاخطیة في تأثیر المتغير المستقل في المتغير التابع سواء في الأجل القصير أو الطويل علاوة على ذلك، یمثل نموذج NARDL أداة قویة لاختبار التكامل بین مجموعة من متغی ارت السلسلة الزمنية في معادلة واحدة. على عكس نماذج التكامل المشترك الأخرى حیث یجب أن یكون ترتیب التكامل للسلسلة الزمنیة المذكورة هو نفسه كذلك، كما یساعد نموذج NARDL على حل مشكلة عدم التجانس باختیار فت ارت الإبطاء المناسبة للمتغیرات. 
     كما یستعمل بشكل أفضل لتحدید علاقات التكامل في العینات الصغیرة ویمكن تطبیقها بغض النظر عما إذا كانت السلاسل مستقرة عند المستوى أو عند الفرق الأول أو مزیج بینهما. غیر أنه لا یمكن تطبیقها في حالة ما إذا كانت السلسلة مستقرة من الفرق الثاني. كما أنه یسمح بكشف التكامل المشترك الخفي. حیث على سبیل المثال قد یكون للصدمة الإیجابیة تأثیر مطلق كیر على المدى القصیر في حین الصدمة السلبية یكون لها تأثیر مطلق كبیر في المدى الطويل. أو العكس.
4-4 الصیغة الریاضیة لنموذج NARDL: 
    تم تقسیم المتغیر المستقل 𝑋𝑋𝑡𝑡 ما بین قیم سالبة −𝑋𝑋𝑡𝑡 وأخرى موجبة +𝑋𝑋𝑡𝑡. وانطلاقا من هذا التقسيم، فإن إدخال كلا المتغيرين في نموذج ARDL سوف ینتج لدینا نموذج NARDL وفق الصيغة الآتية
حیث: +θ−; 𝜃𝜃  تمثل معلومات الأجل الطویل للعلاقة التناظریة في النموذج؛ +𝜋𝜋𝑗𝑗−; 𝜋𝜋𝑗𝑗  تمثل المقد ارت غیر التناظریة في الأجل القصیر. 
      وتتشابه الاختبا ارت التشخیصیة لنموذج NARDL  مع نموذج ARDL، إذ یتم اختبار التكامل المشترك وفق فرضیة العدم ( 0 = −𝜇𝜇 = 𝜌𝜌 = 𝜃𝜃+ = 𝜃𝜃)، فضلا عن اختبار التوزیع الطبیعي لحد الخطأ واستق ارر النموذج بالإضافة إلى اختبا ارت اختلاف التباین واستقلالیة حد الخطأ. 
   یتمیز أسلوب NARDL باختبار إضافي هو اختبار التماثل Symmetry في الأجل الطویل، حیث یتم اختبار فرضیة العدم 0H باستخدام اختبار Wald Test مقابل الفرضیة البدیلة والتي تنص على عدم التماثل (Asymmetry)7. تعطى فرضیة العدم لهذا الاختبار بالصیغة الآتیة: 
 …………………………………………………….…(2)
     كما یمكن استعمال هذا الاختبار في إیجاد اختبار التماثل للمعلمات في الأجل القصیر 8. وفق فرضیة العدم التالیة: 
 …………………………………………………………….….(3)
     من الاختبا ارت التشخیصیة المهمة في نموذج NARDL اختبار مضاعف التأثیر التراكمي الدینامیكي غیر المتماثل (Asymmetric  Cumulative  Dynamic  Multiplier  Effect) في المتغیر التابع الناجم عن التغیر في المتغیر المستقل الموجب ( +x) والمتغیر المستقل السالب (-x) بوحدة واحدة. والذي یعكس تأثیر الصدمات قصیر الأجل في سلوك المتغیر التابع، كما یبین الأثر التناظري لهذه الصدمات في الأجل القصیر.  
ثانیا: خطوات تقدیر نموذج NARDL: 
    بعد الاطلاع على د ارسات قیاسیة في هذا الموضوع 9 یمكن استخلاص أهم الخطوات التي یمر بها الباحث من أجل تقدیر نموذج NARDL وذلك كمایلي: 
1- معالجة متغیارت النموذج: 
الاحصاء الوصفي لمتغی ارت النموذج. 
د ارسة استقرارية متغيرات النموذج. 
2- تقدیر نموذج NARDL: 
اختبار فت ارت الإبطاء المثلى لنموذج NARDL. 
تقدیر نموذج  NARDL بعد تحدید فت ارت الإبطاء المثلى. 
اختبار الحدود لنموذج  NARDL. 
اختبار جودة النموذج: 
- الاختبا ارت التشخیصیة لنموذج NARDL: 
اختبار مضروب لاغرنج للارتباط التسلسلي بین البواقي (BG). 
اختبار عدم ثبات التباین المشروط بالانحدار الذاتي (ARCH). 
اختبار التوزیع الطبیعي للأخطاء العشوائیة (Jarque- Bera). 
اختبار مدى ملائمة تحديد وتصميم النموذج المقدر من حدث الشكل الدالي (Ramsey). 
اختبار التماثل في الأجل الطويل والقصیر. 
اختبار مضاعف التأثیر النراكمي الديناميكي غیر المتماثل. 
اختبار الاستقرر الهیكلي لمعلمات نموذج  NARDL. 
اختبار الأداء التنبؤي لنموذج NARDL. 
6- تقدير معلمات الأجلين القصير والطويل ومعلمة تصحيح الخطأ لنموذج NARDL. 
1-تمهيد
تعد الدراسة الوصفية لبيانات البحث مرحلة أساسية ومحورية في أي مسار تحليلي أو بحثي، إذ تتيح فهما عاما ومبسطا لخصائص البيانات التي تم جمعها، وتبرز أهميتها في قدرتها على تقديم مؤشرات أولية حول توزيع المتغيرات، ومقاييس النزعة المركزية، ودرجات التشتت، إلى جانب الكشف عن أية قيم متطرفة أو غير طبيعية قد تؤثر على مصداقية نتائج التحليل، من خلال هذا النوع من التحليل، يتمكن الباحث من تقييم مدى ملاءمة البيانات لإجراء الاختبارات الإحصائية وتطبيق النماذج الاقتصادية، مما يساهم في اختيار الأدوات التحليلية الأنسب وضمان دقة النتائج، كما تتيح الدراسة الوصفية عرض ملخصات واضحة تسهل تفسير البيانات وتوصيلها بفعالية إلى جمهور القراء أو صناع القرار، الأمر الذي يعزز من وضوح البحث ومصداقيته. 
2- وصف متغيرات الدراسة:
الشكل1: تطور دعم العائلات خلال الفترة 2000-2022
يظهر المنحنى تطور دعم العائلات خلال الفترة الممتدة من سنة 2000 إلى سنة 2022 اتجاها تصاعديا واضحا يعكس تزايدا مستمرا في حجم الإنفاق الاجتماعي الموجه نحو الأسر، فقد بدأ هذا الدعم بقيمة منخفضة نسبيا قدرها 47.33 مليار سنة 2000، ثم شهد نموا تدريجيا خلال السنوات الأولى، ليصل إلى 96.12 مليار في سنة 2005، وهو ما يمثل تقريبا تضاعفا في قيمة الدعم خلال خمس سنوات، ابتداء من سنة 2006، بدأت وتيرة الارتفاع تتسارع بشكل ملحوظ، حيث انتقل الدعم من 136.9 مليار في تلك السنة إلى 176.34 مليار في 2007، ثم شهد قفزة كبيرة في سنة 2008 ليصل إلى 402 مليار، أي بزيادة تجاوزت 128% مقارنة بالسنة التي سبقتها، وهو ما يمكن أن يعكس تحولا في السياسة الاجتماعية أو استجابة لتحديات اقتصادية معينة.
خلال الفترة ما بين 2009 و2015، استقر مستوى الدعم عند معدلات مرتفعة نسبيا، مع بعض التذبذب، حيث تراوح ما بين 318 و492 مليار ورغم هذا التذبذب، فإن المستوى العام للدعم ظل مرتفعا مقارنة بالمراحل السابقة، مما يدل على تثبيت الإنفاق الاجتماعي ضمن أولويات الدولة أما الفترة الممتدة من 2016 إلى 2021، فقد تميزت بنوع من الاستقرار النسبي في قيم الدعم، والتي دارت غالبا حول مستوى 400 إلى 470 مليار، مما يعكس نوعا من التوازن بين الإمكانات المالية والاحتياجات الاجتماعية.
لكن الملاحظ أن سنة 2022 شهدت قفزة جديدة وكبيرة في قيمة الدعم، حيث بلغت 587.69 مليار، بزيادة تناهز 25% مقارنة بسنة 2021، وهي زيادة معتبرة قد تعكس تأثير عوامل اقتصادية كارتفاع معدلات التضخم، أو توسع في برامج الدعم الاجتماعي، أو توجها حكوميا لتوسيع شبكات الحماية الاجتماعية في ظل أزمات اقتصادية أو اجتماعية محتملة ومن خلال هذا المسار الزمني العام، يمكن القول إن منحنى دعم العائلات عرف نموا مطردا ومستمرا، مع مراحل تسارع ملحوظة تزامنت في الغالب مع ظروف استثنائية، مما يشير إلى مرونة في التوجهات المالية وقدرة الدولة على تعديل سياساتها الاجتماعية وفق مقتضيات المرحلة، بما يعزز الاستقرار الاجتماعي ويدعم الفئات الهشة في المجتمع.
الشكل2: تطور دعم العائلات خلال الفترة 2000-2022
الشكل3: تطور دعم المعاشات خلال الفترة 2000-2022

يعكس تطور دعم المعاشات في الفترة الممتدة من سنة 2000 إلى سنة 2022 توجها تصاعديا واضحا في حجم الإنفاق الاجتماعي الموجه نحو المتقاعدين، بما يدل على اهتمام متزايد بتأمين الحياة الكريمة لهذه الفئة، فقد بدأ دعم المعاشات بمستوى منخفض نسبيا في سنة 2000، حيث لم يتجاوز 19.45 مليار، غير أنه تضاعف تقريبا في السنة التالية (2001) ليصل إلى 38.36 مليار، واستمر في الارتفاع التدريجي خلال السنوات التالية، حيث بلغ 76.69 مليار في 2004، وهو ما يمثل نموا يفوق 290% خلال أربع سنوات فقط.
ابتداء من سنة 2005، استمر الاتجاه التصاعدي مع تسجيل بعض التذبذبات الطفيفة، حيث ارتفعت القيمة إلى 90.26 مليار في سنة 2007، ثم إلى 115.74 مليار في 2008، وبلغت 149.25 مليار في 2009، مما يعكس سياسة توسعية في مجال الحماية الاجتماعية خلال تلك المرحلة، ورغم انخفاض طفيف في 2010 و2011، إلا أن القيم ظلت مرتفعة نسبيا، مما يشير إلى نوع من الاستقرار في حجم الدعم.
أما التحول الأبرز في المسار الزمني فقد سجل في سنة 2013، حيث قفز الدعم إلى 257.94 مليار، مقارنة بـ 153.42 مليار في السنة التي سبقتها (2012)، وهي زيادة ضخمة بنسبة تتجاوز 68%، وربما تعود هذه القفزة إلى مراجعات في منظومة التقاعد أو إلى إدراج فئات جديدة ضمن منظومة الدعم، تلت هذه القفزة فترة من الاستقرار النسبي ما بين 2014 و2018، حيث تراوحت قيمة الدعم حول 250 مليار، مع تقلبات طفيفة تشير إلى مرونة في ضبط الإنفاق حسب الإمكانيات المالية والضغوط الاقتصادية.
في السنوات الأخيرة، تحديدا من 2019 إلى 2022، عاد الدعم ليواصل ارتفاعه تدريجيا، حيث بلغ 281.97 مليار في 2019، ثم 288.38 مليار في 2020، ليصل إلى 290.75 مليار في 2021، مسجلا في سنة 2022 أعلى قيمة له على الإطلاق بـ 332.53 مليار، وهو ما يعكس تزايد أعباء المعاشات إما بسبب شيخوخة السكان أو توسع التغطية الاجتماعية أو ارتفاع كلفة المعيشة، مما اضطر الدولة إلى زيادة مخصصات هذا الباب من الميزانية.
بشكل عام، يظهر تحليل هذا الجدول التزاما تدريجيا من الدولة بتعزيز دعم المتقاعدين وضمان حد أدنى من الدخل لهم، مع مرونة نسبية في التعامل مع المتغيرات الاقتصادية والاجتماعية، وهو ما يعكس بعدا إنسانيا واجتماعيا في السياسات العمومية.
الشكل4: تطور دعم الصحة خلال الفترة 2000-2022
يعكس تطور دعم الدولة لقطاع الصحة خلال الفترة الممتدة من سنة 2000 إلى سنة 2022 التوجه المتصاعد نحو تعزيز الإنفاق العمومي في المجال الصحي، بما يعكس إدراكا متزايدا لأهمية هذا القطاع في ضمان الرفاه الاجتماعي وتعزيز التنمية البشرية ففي بداية الألفية، كان مستوى الدعم الصحي متواضعا نسبيا، حيث بلغت قيمته 33.29 مليار في سنة 2000، وارتفع تدريجيا في السنوات اللاحقة ليصل إلى 60.02 مليار في 2003 و63.4 مليار في 2004، مما يشير إلى نمو بطيء ولكن مستقر في بداية العقد.
من سنة 2005 إلى 2007، شهد الدعم الصحي زيادات طفيفة، حيث تراوح بين 60.44 مليار و79.62 مليار، ولكن التحول الجذري بدأ يظهر سنة 2008، حيث ارتفع الدعم بشكل لافت إلى 151.73 مليار، أي أكثر من ضعف الدعم المسجل في السنة التي سبقتها، استمر هذا الاتجاه التصاعدي في 2009 (176.95 مليار) و2010 (199.28 مليار)، وهو ما يمكن تفسيره بإصلاحات أو توسعات في البنية التحتية الصحية أو زيادة الإنفاق على المعدات والموارد البشرية.
أما القفزة الكبرى فقد سجلت في سنة 2011، حيث بلغ دعم الصحة 367.82 مليار، في حين ارتفع إلى ذروته سنة 2012 بـ395.85 مليار، وهو ما يعكس تحولات عميقة في السياسة الصحية للدولة، ربما نتيجة لتنامي المطالب الاجتماعية أو محاولة تحسين مؤشرات الصحة العمومية، غير أن هذا الارتفاع لم يستمر على نفس المنوال، إذ تراجع الدعم إلى 263.71 مليار في 2013، قبل أن يعاود الارتفاع التدريجي في السنوات التالية.
ابتداء من سنة 2014 إلى غاية 2022، حافظ دعم الصحة على مستوى مرتفع نسبيا، يتراوح بين 320 و360 مليار، مع تسجيل نوع من الاستقرار والتذبذب البسيط، حيث بلغ في 2015 نحو 325.4 مليار، و321.34 مليار في 2016، ثم شهد زيادات طفيفة في السنوات اللاحقة ليصل إلى 336.87 مليار في 2019، و354.68 مليار في 2020، ثم 361.12 مليار في 2022، هذه القيم تعكس نوعا من الثبات النسبي في الالتزامات المالية للدولة تجاه القطاع الصحي، رغم التحديات الاقتصادية.
يمكن اعتبار هذه الأرقام دليلا على التحول الاستراتيجي نحو إعطاء أولوية للقطاع الصحي في السياسات العمومية، خاصة بعد سنة 2010، وهو ما قد يكون مرتبطا بتوسع التغطية الصحية، تحسين البنية التحتية، أو مواجهة تحديات صحية جديدة، مثل انتشار الأمراض المزمنة أو الكوارث الصحية، وعليه، فإن هذا التطور في دعم الصحة يعد مؤشرا مهما على التزام الدولة بتحسين جودة الخدمات الصحية وضمان الوصول العادل إليها لجميع المواطنين.
الشكل5: تطور دعم المجاهدين خلال الفترة 2000-2022
يعكس تطور الإنفاق العمومي المخصص لدعم فئة المجاهدين في الجزائر خلال الفترة الممتدة من سنة 2000 إلى سنة 2022 توجها ثابتا نسبيا من الدولة للحفاظ على التزاماتها التاريخية والاجتماعية تجاه هذه الفئة التي تمثل رمزية وطنية عالية، يظهر الجدول أن بداية الألفية شهدت مستويات متقلبة نسبيا من الدعم، إذ سجل مبلغ 60.42 مليار دينار سنة 2000، ثم انخفض إلى 56.83 مليار سنة 2001، ليعرف قفزة معتبرة في سنة 2002 مسجلا 78.06 مليار دينار، تلتها بعض التذبذبات الطفيفة خلال السنوات اللاحقة.
من سنة 2003 إلى 2007، تراوح الدعم بين 63.35 و82.08 مليار دينار، وهو ما يشير إلى نوع من الاستقرار النسبي في الإنفاق، مع تسجيل زيادات تدريجية تبرز التزاما متواصلا من الدولة تجاه هذه الفئة أما سنة 2008، فقد عرفت ارتفاعا واضحا في الدعم ليصل إلى 108.28 مليار، وتواصل المنحى التصاعدي في السنوات التالية، إذ بلغ 124.05 مليار في 2010، وارتفع إلى 142.99 مليار في 2012.
وقد تميزت الفترة من 2013 إلى 2016 بزيادات معتبرة في حجم الدعم، حيث وصل إلى 171.94 مليار سنة 2013، وبلغ ذروته تقريبا في سنة 2015 بـ198.22 مليار دينار، يعكس هذا التوجه رغبة السلطات العمومية في تعزيز الامتيازات الاجتماعية والمادية الممنوحة للمجاهدين وذوي الحقوق، ربما نتيجة مراجعات قانونية أو توسعات في الفئة المستفيدة.
ابتداء من سنة 2016، استقر مستوى الدعم حول عتبة 197 إلى 200 مليار دينار، مع تراجع طفيف خلال السنوات الأخيرة، حيث بلغ 198.4 مليار في 2020، ثم 196.12 مليار في 2022، يشير هذا الاستقرار إلى بلوغ الدعم مرحلة نضج من حيث الهيكلة المالية، مع الحفاظ على مستوى مرتفع نسبيا مقارنة ببداية الألفية، وهو ما يعكس سياسة اجتماعية واضحة تعطي أهمية لفئة المجاهدين باعتبارها جزءا من الذاكرة الوطنية.
بشكل عام، يبرز هذا التحليل التزاما طويل الأمد من الدولة الجزائرية بدعم المجاهدين، ليس فقط من منطلق الوفاء التاريخي، بل أيضا في إطار سياسة عدالة اجتماعية تثمن دورهم في التحرير الوطني، مع تكييف هذا الدعم تدريجيا وفقا للمعطيات الاقتصادية والديمغرافية
الشكل6: تطور دعم المعوزين المعاقين وذوي المداخيل الضعيفة خلال الفترة 2000-2022
يعكس تطور الإنفاق العمومي في الجزائر الموجه لفئة المعوزين، المعاقين، وذوي الدخل المحدود خلال الفترة الممتدة من سنة 2000 إلى سنة 2022 التزاما اجتماعيا واضحا من الدولة تجاه الشرائح الأكثر هشاشة في المجتمع في بداية الألفية، اتسم هذا الدعم بمستويات متواضعة، حيث بلغ 34.16 مليار دينار سنة 2000، ثم سجل زيادات تدريجية طفيفة في السنوات الثلاث اللاحقة، ليصل إلى 55.36 مليار دينار في عام 2003، غير أن سنة 2004 شهدت تراجعا نسبيا إلى 46.42 مليار دينار، تلاه ارتفاع طفيف في 2005 إلى 50.61 مليار.
ابتداء من سنة 2006، سجلت قفزة كبيرة في حجم الدعم، حيث بلغ 91.41 مليار دينار، وهو ما يمكن تفسيره بإصلاحات اجتماعية أو توسيع للفئات المستفيدة، رغم أن سنة 2007 شهدت تراجعا نسبيا إلى 75.93 مليار، إلا أن المنحى العام ظل تصاعديا، ليصل الدعم إلى 100.51 مليار في 2008، ثم 115.41 مليار في 2009، مما يعكس توجها استراتيجيا لتعزيز الحماية الاجتماعية في سياق ارتفاع أسعار النفط آنذاك.
عرف الدعم قفزة نوعية أخرى في سنة 2011، حيث بلغ 185.53 مليار دينار، واستمر على هذا المنوال في سنة 2012 بـ185.86 مليار، وهي مستويات قياسية مقارنة بالسنوات السابقة، خلال الفترة 2013–2016، سجلت تقلبات طفيفة في حجم الدعم، حيث تراوح بين 163.22 و178.66 مليار دينار، ما يعكس محاولة الدولة الحفاظ على توازن بين الاستمرارية في الدعم وقيود الميزانية.
أما في السنوات الأخيرة (2017–2022)، فقد شهد الدعم تقلبات طفيفة مع تسجيل مستويات مرتفعة نسبيا، إذ بلغ 153 مليار دينار في 2019، وارتفع إلى 168.37 مليار في 2020، ثم قفز بشكل كبير إلى 209.57 مليار في 2021، قبل أن يسجل استقرارا نسبيا عند 207.6 مليار دينار في 2022، هذا التطور الأخير يعبر عن استجابة الدولة للضغوط الاجتماعية والاقتصادية التي خلفتها جائحة كوفيد-19، خاصة على الفئات الضعيفة، مع تعزيز آليات الدعم لتقليص آثار الأزمة وتحقيق العدالة الاجتماعية.
في المجمل، يظهر مسار هذا الإنفاق تصاعدا عاما يعكس سياسة اجتماعية تهدف إلى تحسين أوضاع الفئات الهشة، مع مراعاة التحولات الاقتصادية والظروف الاجتماعية التي عرفتها البلاد، مما يعزز من دور الدولة كفاعل مركزي في الحماية الاجتماعية ومحاربة الفقر والتهميش.
الشكل6: تطور دعم الإنفاق العمومي الموجه للدعم الإجتماعي خلال الفترة 2000-2022

يعكس تطور الإنفاق العمومي الموجه للدعم الاجتماعي في الجزائر خلال الفترة الممتدة من سنة 2000 إلى سنة 2022 التزام الدولة القوي بسياسة الرعاية الاجتماعية، وسعيها الدائم إلى تعزيز شبكات الحماية الاجتماعية لفائدة مختلف الشرائح السكانية، خصوصا الفئات الهشة، فقد انطلق حجم الدعم في سنة 2000 من مستوى 262.4 مليار دينار، ليسجل زيادة متواصلة خلال السنوات الأولى، إذ بلغ 314.99 مليار دينار في 2001 وارتفع إلى 367.66 مليار في 2002، مما يعكس توجها تدريجيا نحو تقوية الدعم العمومي.
واستمر هذا النسق التصاعدي خلال الفترة 2003–2007، حيث ارتفع الدعم من 416.31 مليار دينار إلى 708.57 مليار دينار، وهو ما يمكن ربطه بالتحسن النسبي في المداخيل البترولية وارتفاع أسعار النفط، ما مكن الدولة من توسيع تدخلها الاجتماعي، وسجل الدعم الاجتماعي قفزة نوعية سنة 2008، ليبلغ 1164.04 مليار دينار، ثم واصل الارتفاع في 2009 (1207.86 مليار) و2010 (1239.26 مليار)، في ظل تركيز السياسات العمومية على تدعيم القدرات الشرائية وتحسين مستوى معيشة المواطنين.
عرفت سنة 2011 طفرة استثنائية، حيث تجاوز الدعم الاجتماعي عتبة 2000 مليار دينار مسجلا 2065.07 مليار، وهو أعلى مستوى في تلك المرحلة، ويعزى هذا الارتفاع إلى الإجراءات الاجتماعية الاستثنائية التي اتخذت إثر التحولات السياسية والإقليمية في المنطقة، والتي فرضت على الدولة تعزيز سياستها الاجتماعية لامتصاص التوترات وتحقيق الاستقرار الداخلي، ورغم تراجع طفيف في الدعم خلال 2012 (1763.67 مليار) و2013 (1521.73 مليار)، إلا أن هذا التراجع لم يكن كبيرا، وتم تعويضه لاحقا من خلال انتعاش في حجم الإنفاق الاجتماعي.
شهدت الفترة الممتدة من 2014 إلى 2022 مستوى مرتفعا وثابتا نسبيا من الدعم الاجتماعي، رغم الأزمات الاقتصادية وتراجع عائدات المحروقات، ففي سنة 2015 بلغ الدعم 1830.51 مليار دينار، وواصل نسقه التصاعدي تدريجيا ليصل إلى 1927.5 مليار في 2021، و1942 مليار دينار في 2022، ويعكس هذا الاستقرار النسبي خلال السنوات الأخيرة مرونة السياسة الاجتماعية الجزائرية وقدرتها على التكيف مع الأوضاع المالية الصعبة، خصوصا في ظل تداعيات جائحة كوفيد-19، حيث حافظت الدولة على مستويات مرتفعة من الدعم لضمان الأمن الاجتماعي وتقليل حدة الفقر والتهميش.
بصفة عامة، يبرز هذا المسار التصاعدي في الدعم الاجتماعي مدى التزام الجزائر بتعزيز العدالة الاجتماعية، والحرص على تحقيق التوازن بين مقتضيات التسيير المالي والتزاماتها الاجتماعية، مما يجعل من سياسة الدعم إحدى الركائز الأساسية للسياسات العمومية في البلاد.

الشكل7: تطور معدلات التضخم خلال الفترة 2000-2022

يعكس تطور معدل التضخم في الجزائر خلال الفترة الممتدة من سنة 2000 إلى سنة 2022 تقلبات اقتصادية لافتة تأثرت بعدة عوامل داخلية وخارجية، شملت أسعار المواد الأساسية، وسياسات الإنفاق العام، وتقلبات أسعار النفط، إضافة إلى الأزمات الاقتصادية العالمية فقد سجل معدل التضخم في مطلع الألفية (سنة 2000) مستوى منخفضا بلغ 0.3%، ما يعكس استقرارا نسبيا في الأسعار في تلك المرحلة، لكن هذا المعدل عرف ارتفاعا ملحوظا في السنة التالية (2001) ليبلغ 4.2%، قبل أن يتراجع مجددا إلى 1.4% في 2002، مما يدل على حساسية الأسعار للتغيرات في السياسة النقدية والمالية.
في الفترة الممتدة من 2003 إلى 2008، تراوح معدل التضخم بين 3.7% و4.9%، مع تسجيل ارتفاع ملحوظ في سنة 2008 (4.9%)، وهو ما يمكن ربطه بالأزمة المالية العالمية التي أثرت على سلاسل التوريد وأسعار المواد الغذائية، ما انعكس على الاقتصاد الوطني، وشهدت السنوات اللاحقة ارتفاعا تدريجيا في التضخم، حيث بلغ 5.7% في 2009 و4.5% في 2011، مما يعكس استمرار الضغوط التضخمية، لا سيما نتيجة ارتفاع الإنفاق العام الموجه للدعم الاجتماعي خلال تلك الفترة.
وشهدت سنة 2012 قفزة استثنائية في معدل التضخم بلغت 8.9%، وهي أعلى نسبة خلال تلك المرحلة، ويرجح أن يكون ذلك نتيجة لارتفاع أسعار المواد الاستهلاكية الأساسية وضعف فعالية آليات ضبط السوق بعد هذا الارتفاع، عرف التضخم بعض الاستقرار النسبي في الفترة ما بين 2013 و2015، حيث تراوح بين 2.9% و4.8%، قبل أن يعاود الارتفاع مجددا في 2016 (6.4%) و2017 (5.6%)، ما يعكس استمرار الضغوط الناتجة عن تقلب أسعار السلع، خاصة في ظل تراجع قيمة الدينار وتباطؤ النشاط الاقتصادي.
أما في السنوات الأخيرة، وتحديدا من 2018 إلى 2022، فقد شهد معدل التضخم تقلبات جديدة، فبعد انخفاض نسبي في 2019 (2%) و2020 (2.4%)، عاد التضخم إلى الارتفاع بقوة سنة 2021 مسجلا 7.2%، ثم بلغ ذروته في سنة 2022 عند 9.3%، وهو أعلى مستوى في الفترة كلها، ويرتبط هذا التصاعد الحاد بمجموعة من العوامل، أبرزها تداعيات جائحة كوفيد-19، وتعطل سلاسل الإمداد العالمية، وتراجع القدرة الشرائية، إضافة إلى تدفقات نقدية كبيرة في إطار الدعم الاجتماعي والإنفاق العمومي، دون توافر عرض كاف من السلع والخدمات.
بوجه عام، يظهر تحليل بيانات التضخم في الجزائر خلال هذه الفترة أن الاقتصاد الوطني ظل عرضة لتقلبات متعددة، وأن التحكم في التضخم ظل تحديا مستمرا، يتطلب سياسات نقدية ومالية منسقة، وتحسينا في إنتاجية القطاعات الحيوية، خاصة الفلاحة والصناعة، لتقليص الاعتماد على الواردات وتفادي الضغوط السعرية المستوردة.
إختبار فرضيات الدراسة:
في هذه المرحلة، سنعتمد على منهجية النمذجة الديناميكية الغير خطية (NARDL) لاختبار العلاقة طويلة الأمد بين المتغير التابع وعدد من المتغيرات المستقلة، لما توفره من مرونة في التعامل مع متغيرات من رتب تكامل مختلفة، تشمل المنهجية اختبار حدود بيساران للكشف عن التوازن طويل المدى، يليها بناء نموذج تصحيح الخطأ (ECM) لتقدير سرعة التعديل نحو هذا التوازن.
4-1-2- اختبار عدم التماثل 
تشير نتائج اختبار والد (Wald Test) إلى أن قيمة إحصائية F بلغت (6.692) بدرجة حرية (1,14)، وباحتمالية تقدر بـ(0.021)، كما بلغت قيمة اختبار مربع كاي (Chi-square) لنفس الفرضية (6.692) باحتمالية (0.0097)، وهي أقل من مستوى الدلالة المعتمد 0.05، تدل هذه النتائج على رفض فرضية العدم القائلة بوجود تماثل بين تأثيرات التغيرات الإيجابية والسلبية للمتغيرات المستقلة على المتغير التابع، وقبول الفرضية البديلة التي تنص على وجود عدم تماثل معنوي في العلاقة المدروسة.
وبناء على ذلك، فإن النموذج الأنسب لتحليل العلاقة بين المتغيرات في هذه الحالة هو نموذج NARDL (الانحدار الذاتي غير الخطي للإبطاء الموزع)، باعتباره الأكثر قدرة على التقاط الفروقات بين تأثيرات الارتفاع والانخفاض في المتغيرات التفسيرية، والتي ثبتت دلالتها إحصائيا من خلال هذا الاختبار وبالتالي، يوصى باعتماد نموذج NARDL في التحليل لضمان الحصول على تقديرات دقيقة وواقعية لطبيعة العلاقة.
4-2-1 اختبار النموذج:
سنعرض بشكل موجز المتغيرات الأساسية للدراسة، مع تحديد طبيعتها (مستقلة، تابعة، أو متحكم فيها)، واختيار النموذج التحليلي المناسب، ثم توضيح الفرضيات الرئيسية، وآلية ترميز المتغيرات لتسهيل المعالجة الإحصائية وضمان دقة النتائج.
4-2-1-1: وصف النموذج
قمنا بصياغة النموذج التالي INF=f(SUH, SUF, SUP, SHEAL, SUM, SUN, SU,) ويمكن ترجمته اقتصاديا وكميا على الشكل التالي:
ويمثل كل رمز ما يلي: 
INF : ويمثل معدل التضخم 
SUH:   تمثل دعم السكن
SUFSU: يمثل دعم العائلات
SUP:   يمثل دعم المعاشات
SHEAL :  يمثل دعم الصحة
SUM :  يمثل دعم المجاهدين
SUN :  يمثل دعم المعوزين، المعاقين، وذوي المداخل الضعيفة
SU : يمثل الدعم الاجتماعي
4-2-1-2: : دراسة استقرارية النموذج
	تعد دراسة استقرارية متغيرات الدراسة خطوة محورية في البحوث التي تعتمد التحليل الزمني، خاصة عند بناء النماذج غير الخطية، إذ تساهم في الكشف عن مدى ثبات السلاسل الزمنية وخلوها من الاتجاهات العشوائية والانحرافات التي قد تؤدي إلى نتائج تحليلية مضللة، تأكيد استقرارية المتغيرات يعد شرطا أساسيا لتطبيق النماذج غير الخطية بشكل سليم، حيث يؤدي عدم الاستقرار إلى علاقات زائفة (Spurious Relationships) ويقلل من دقة التقديرات، كما أن التحقق من الاستقرارية يسمح باختيار النموذج المناسب للبيانات، سواء كان خطيا أو غير خطي، ويعزز من موثوقية النتائج والتوصيات المستخلصة وفي هذا السياق، يقدم الجدولان التاليان نتائج اختبار الاستقرارية باستخدام اختباري ديكي-فولر وفليبس-بيرون لجذر الوحدة، وذلك لتقييم مدى ملاءمة المتغيرات للدراسة وتحديد خصائصها الزمنية بدقة، ويوضح الجدولين التاليين نتائج إختبار الاستقرارية من خلال اختبار ديكي فولار وفليبس وبييرون لجذر الوحدة لمتغيرات الدراسة
الجدول رقم1: إختبار السكون للمتغيرات التابعة والمستقلة
تشير نتائج اختباري ديكي-فولر الموسع (ADF) وفليبس-بيرون (PP) لجذر الوحدة إلى أن جميع المتغيرات محل الدراسة، باستثناء التضخم (INF) تحت بعض الفروض، غير مستقرة عند المستوى (Level) سواء بوجود ثابت فقط أو بوجود ثابت واتجاه عام أو بدون كليهما، حيث تجاوزت قيم الاحتمال (p-values) في أغلب الحالات عتبة الدلالة الإحصائية (عادة 0.05)، مما يعني عدم رفض فرضية العدم القائلة بوجود جذر وحدة، وبالتالي عدم استقرارية السلاسل الزمنية لهذه المتغيرات على سبيل المثال، سجل متغير دعم الصحة (SUH) قيم احتمالية مرتفعة في كلا الاختبارين تتراوح بين 0.16 و0.33، مما يؤكد عدم استقراريته، وينطبق الأمر ذاته على باقي المتغيرات مثل دعم العائلات (SUF)، دعم المعاشات (SUP)، دعم المجاهدين (SUM)، دعم المعوزين (SUN)، ودعم الفئات الخاصة (SHEAL).
أما بالنسبة لمتغير التضخم (INF)، فقد أظهر سلوكا مختلفا، حيث سجل قيمة احتمالية مقبولة نوعا ما في اختبار فليبس-بيرون دون ثابت أو اتجاه (0.0519)، وكذلك في اختبار ديكي فولر بوجود ثابت فقط (0.0561)، مما قد يشير إلى استقرارية ضعيفة أو حدودية لهذا المتغير تحت بعض الفرضيات، ويستدعي إجراء اختبار الاستقرارية عند الفرق الأول (First Difference) للتأكد النهائي.
بالتالي، توضح النتائج أن المتغيرات الاقتصادية والاجتماعية في هذه الدراسة تعد غير مستقرة عند المستوى، مما يفرض استخدام منهجيات تحليلية تأخذ بعين الاعتبار هذا الطابع، مثل نماذج التكامل المشترك (Cointegration) أو نماذج الانحدار الذاتي للفجوات الزمنية (ARDL) عند الحاجة إلى تحليل العلاقات السببية أو التنبؤية بينها.
الجدول رقم2: اختبار استقرارية السلسلة بعد الفروق من درجة الأولى
يظهر الجدول نتائج اختبار استقرارية متغيرات الدراسة بعد أخذ الفروق من الدرجة الأولى، وذلك باستخدام كل من اختبار ديكي-فولر الموسع (ADF) واختبار فليبس-بيرون (PP)، عبر ثلاثة نماذج مختلفة: بدون ثابت واتجاه، مع ثابت فقط، ومع كل من الثابت والاتجاه العام.
تشير القيم الإحصائية المحسوبة (P-Values) في كلا الاختبارين إلى أن جميع المتغيرات أصبحت مستقرة بعد أخذ الفرق الأول، إذ جاءت جميع القيم أقل من المستوى المعنوي 0.05، وهو ما يدل على رفض فرضية وجود جذر وحدة (أي وجود لا استقرارية) لصالح الفرض البديل القائل بأن السلاسل الزمنية أصبحت مستقرة.
فعلى سبيل المثال، المتغير SUH أظهر دلالة معنوية عالية في جميع النماذج عبر كلا الاختبارين، حيث سجلت القيم مستويات قريبة من الصفر، ما يؤكد استقراره بعد الفرق، وينطبق الأمر ذاته على المتغيرات SUF، SUP، SHEAL، SUM، SUN، وSU، والتي أظهرت كلها نتائج مماثلة، وإن تباينت قليلا في القيم بين النماذج المختلفة، إلا أنها بقيت جميعها دون عتبة الدلالة الإحصائية المقبولة.
أما بالنسبة لمتغير INF (التضخم)، فقد أظهر بدوره استقرارية واضحة بعد الفرق الأول، لا سيما في اختبار ديكي-فولر، حيث جاءت القيم قريبة من الصفر تماما في كل النماذج، وهو ما يعزز من موثوقية إدخاله في نماذج التحليل اللاحقة.
بناء عليه، تؤكد هذه النتائج أن جميع المتغيرات محل الدراسة تحقق شرط الاستقرارية عند الفروق من الدرجة الأولى، ما يتيح استخدامها بشكل آمن في النماذج الاقتصادية غير الخطية، ويجنب الباحث خطر الوقوع في الارتباطات الزائفة التي قد تؤثر سلبا على مصداقية الاستنتاجات التحليلية.
4-2-1-3:  تحديد فترات الإبطاء:
تعد عملية تحديد فترات الإبطاء (Lag Length Selection) خطوة جوهرية في بناء نموذج الانحدار الذاتي للفجوات الزمنية الموزعة غير الخطية (NARDL)، لما لها من تأثير مباشر على دقة تمثيل العلاقات غير الخطية والديناميكية بين المتغيرات محل الدراسة، إذ يتيح اختيار الفترات المثلى للباحث القدرة على التقاط الفروقات في التأثيرات الإيجابية والسلبية على المديين القصير والطويل بشكل أكثر دقة، مما يعزز من مصداقية النموذج ويقلل من مشكلات مثل الارتباط الذاتي وضعف القدرة التفسيرية وبالتالي، فإن تحديد فترات الإبطاء يعد خطوة حاسمة لضمان سلامة النموذج القياسي NARDL وفعاليته في تفسير الظواهر الاقتصادية بشكل واقعي وغير خطي.
الجدول رقم3: تحديد فترة الإبطاء المثلى المختارة
تشير نتائج اختبار فترة الإبطاء المثلى لنموذج NARDL إلى أن النموذج اختار فترة الإبطاء (2,0,0,0,0,0,0,0,0,0,0,0,0)، ما يعني أن فقط المتغير الأول في النموذج، وهو غالبا المتغير التابع، تم تضمينه بإبطاء قدره درجتان، في حين أن بقية المتغيرات المستقلة – وهي: SUF، SUP، SHEAL، SUM، SUN، SU، وSUH – لم تدرج بأي فترات إبطاء، تعكس هذه النتيجة أن تأثير المتغير التابع على نفسه يمتد إلى فترتين سابقتين، بينما التأثيرات الآنية لباقي المتغيرات كافية لتفسير التغيرات في النموذج، دون الحاجة لإدراج تأخيرات إضافية.
وتعتبر هذه الخطوة أهمية خاصة ضمن تحليل الاستقرارية، إذ إن تحديد فترات الإبطاء المثلى بدقة يساهم في تجنب مشكلة الارتباط الذاتي (Autocorrelation) ويوفر تقديرات أكثر موثوقية، كما يعد ذلك مؤشرا على طبيعة العلاقة الديناميكية بين المتغيرات، وهو أمر حاسم في النماذج غير الخطية مثل NARDL، التي تفترض تفاعلات غير متماثلة بين المتغيرات عبر الزمن بالتالي، فإن اختيار هذه البنية من فترات الإبطاء يعكس ملاءمة البيانات للنموذج واستجابتها بشكل متوازن للصدم الاقتصادية، مما يدعم بناء نموذج تحليلي مستقر وفعال 
4-2-1-4:  اختبار حدود التكامل المشترك 
يعد اختبار حدود التكامل المشترك في نموذج NARDL من الأدوات المحورية لتحليل العلاقات طويلة الأمد بين المتغيرات الاقتصادية، خاصة عندما تكون هذه العلاقات غير متماثلة بطبيعتها، تكمن أهمية هذا الاختبار في قدرته على التحقق من وجود علاقة توازنية مستقرة بين المتغيرات، حتى وإن اختلفت درجات تكاملها بين I(0) وI(1)، وهو ما يضفي مرونة أكبر على التحليل مقارنة بالنماذج التقليدية، كما يتيح نموذج NARDL إمكانية التمييز بين تأثيرات الصدمات الإيجابية والسلبية على المتغير التابع، مما يعزز من دقة فهم الديناميكيات الاقتصادية عبر الأجلين القصير والطويل وعليه، فإن اختبار حدود التكامل داخل هذا النموذج يعد خطوة أساسية لضمان مصداقية النتائج وتمكين الباحث من بناء نموذج يعكس الواقع الاقتصادي بصورة أكثر واقعية وعمقا.
الجدول رقم4: نتائج إختبار حدود التكامل المشترك
يوضح جدول نتائج اختبار حدود التكامل المشترك لنموذج NARDL نتائج اختبار F-statistic الذي يستخدم للتحقق من وجود علاقة توازنية طويلة الأجل بين المتغيرات، وذلك في سياق دراسة الاستقرارية وبناء النماذج غير الخطية، تشير قيمة اختبار F المحسوبة البالغة 6،0708 إلى أنها تتجاوز بوضوح القيم الحدية العليا عند جميع مستويات الدلالة (10%، 5%، 2،5%، و1%)، حيث أن أكبر حد عند مستوى 5% مثلا هو 3،04، وهي أقل بكثير من القيمة المحسوبة،
بناء عليه، يمكن رفض الفرضية العدمية لغياب التكامل المشترك، والقبول بوجود علاقة توازنية طويلة الأجل بين المتغيرات المدروسة في النموذج، وتعد هذه النتيجة مؤشرا إيجابيا على استقرارية العلاقة بين المتغيرات عبر الزمن، مما يعزز من صلاحية النموذج المستخدم ويدعم إمكانية استخدامه للتفسير والتحليل والتنبؤ ضمن إطار غير خطي.
 4-2-2- تقدير نموذج الانحدار الذاتي ذي الفجوات الزمنية المتباطئة الموزعة
يعد نموذج NARDL أداة متقدمة لتحليل العلاقات غير الخطية بين المتغيرات الاقتصادية على المدى القصير والطويل، حيث يميز بين التأثيرات الإيجابية والسلبية، مما يوفر فهما أدق للسلوك الاقتصادي، يتميز بمرونته في التعامل مع بيانات متكاملة من درجات مختلفة، واختياره لفترات التأخير المثلى، ما يجعله مناسبا لتحليل الظواهر المعقدة وبناء نماذج أكثر واقعية وموثوقية لدعم اتخاذ القرار
4-2-2-1: تقدير العلاقة قصيرة الأجل في إطار نموذج تصحيح الخطأ 
يعد نموذج الانحدار الذاتي غير الخطي للإبطاء الموزع (NARDL) من النماذج الحديثة والفعالة في تحليل العلاقة قصيرة الأجل بين المتغيرات الاقتصادية، خاصة عندما تكون العلاقة غير متناظرة، يسمح هذا النموذج بتقدير التأثيرات المختلفة للزيادات والانخفاضات في المتغيرات المستقلة على المتغير التابع، مع الأخذ في الاعتبار وجود توازن طويل الأجل تم إثباته مسبقا من خلال اختبار حدود التكامل المشترك.
في إطار NARDL، يتم دمج آلية تصحيح الخطأ (ECM) التي تبرز مدى استجابة المتغيرات لأي انحراف عن العلاقة التوازنية طويلة الأجل، مما يعكس سرعة تعديل النظام وعودته إلى الاستقرار، ويعتبر هذا جانبا أساسيا لفهم التغيرات المؤقتة والديناميكيات الاقتصادية قصيرة الأجل، خاصة في البيئات التي تتسم بعدم اليقين والتقلبات المتكررة بذلك، يوفر نموذج NARDL فهما أدق لتفاعل المتغيرات الاقتصادية على المدى القصير، ويدعم بشكل فعال عملية اتخاذ القرار وصياغة السياسات المالية والاقتصادية المبنية على استجابات آنية وواقعية.
الجدول رقم5: تقدير معلمات النموذج قصير الأجل
يعرض الجدول نتائج تقدير نموذج الانحدار غير الخطي للانطواء الديناميكي (NARDL)، حيث تم تحليل تأثير عدة متغيرات على المتغير التابع، مع التركيز على قيمة المعامل، الخطأ المعياري، اختبار t، وقيمة الاحتمال (p-value) لكل متغير.
أولا، نجد أن الثابت (C) له قيمة معامل موجبة وكبيرة تبلغ حوالي 10.23، مع قيمة احتمال (p=0.0268) أقل من مستوى الدلالة 0.05، مما يشير إلى أن الثابت له تأثير معنوي وإحصائي مهم في النموذج، وهو يمثل القيمة الأساسية للمتغير التابع عند عدم وجود تأثير للمتغيرات المستقلة.
ثانيا، المتغير INF(-1)* الذي يمثل القيمة المتأخرة لمتغير التضخم (أو مؤشر التضخم) له معامل سلبي (-0.993) مع اختبار t سلبي (-2.51) وقيمة احتمال تبلغ 0.0541، وهي قريبة من مستوى الدلالة 0.05، مما يشير إلى أن لهذا المتغير تأثير سلبي هام تقريبا على المتغير التابع، ويمكن اعتباره مهما عند مستوى دلالة أقل قليلا، مما يدل على أن ارتفاع التضخم في الفترة السابقة يميل إلى خفض المتغير التابع في هذه الفترة.
أما بالنسبة لمتغيرات SHEAL_POS وSHEAL_NEG، والتي تمثل التأثيرات الإيجابية والسلبية على التوالي لمتغير SHEAL، فإن كلاهما يظهر معاملات صغيرة وقيم احتمالات مرتفعة (0.2223 و0.3836 على التوالي)، مما يدل على عدم وجود تأثير معنوي إحصائي لهما على المتغير التابع.
       بنفس الطريقة، تظهر المتغيرات SUF_POS وSUF_NEG وSUH_POS وSUH_NEG، وكذلك SUM_NEG وSUN_NEG وSUP_POS وSUP_NEG، معاملات غير معنوية (p-values أكبر من 0.1 في الغالب)، مما يشير إلى أن هذه المتغيرات لا تؤثر بشكل مباشر ومهم على المتغير التابع في هذا النموذج.
ومع ذلك، هناك بعض المتغيرات ذات التأثيرات الإيجابية والسلبية التي تظهر تأثيرات معنوية، مثل SUM_POS وSUN_POS. حيث SUM_POS له معامل سلبي (-0.1935) مع قيمة احتمال 0.0319، ما يدل على تأثير سلبي معنوي على المتغير التابع، بينما SUN_POS له معامل إيجابي (0.1653) مع قيمة احتمال 0.0228، مما يعني أن له تأثير إيجابي معنوي، وهذا يعكس أن التغيرات الإيجابية في SUN_POS ترفع المتغير التابع، بينما التغيرات الإيجابية في SUM_POS تخفضه.
أخيرا، المتغير D(INF(-1)) الذي يمثل التغير في التضخم المتأخر له تأثير غير معنوي عند مستوى الدلالة المعتاد (p=0.3403).
ومنه يمكننا استنتاج أن النموذج يبين وجود تأثيرات معنوية لعدد محدود من المتغيرات على المتغير التابع، وبخاصة التضخم المتأخر INF(-1)*، وبعض التأثيرات الإيجابية والسلبية الخاصة بـ SUM_POS وSUN_POS بينما باقي المتغيرات لا تظهر تأثيرا معنويا خلال الفترة المعنية، هذا يشير إلى أهمية التركيز على متغيرات التضخم والتغيرات في بعض المكونات الإيجابية والسلبية ضمن النموذج لفهم الديناميات المؤثرة على المتغير التابع بشكل أدق.
4-2-2-2  تقدير العلاقة طويلة الأجل:
يعد نموذج الانحدار الذاتي غير الخطي للإبطاء الموزع (NARDL) أداة مهمة لدراسة العلاقة قصيرة الأجل بين المتغيرات الاقتصادية، حيث يميز بين التأثيرات الإيجابية والسلبية غير المتماثلة، يساعد هذا النموذج على فهم كيفية استجابة المتغيرات للصدمات المؤقتة بشكل غير خطي، ويوفر لصانعي السياسات معلومات دقيقة عن التأثيرات الفورية والمتفاوتة للقرارات الاقتصادية بذلك، يعتبر تقدير العلاقة قصيرة الأجل في NARDL خطوة أساسية لفهم الديناميكية الاقتصادية وتمهيدا لتحليل العلاقات طويلة الأجل بدقة أكبر.
الجدول رقم6: تقدير معلمات النموذج طويل الأجل
يعرض الجدول نتائج نموذج NARDL لتحليل العلاقة بين المتغير التابع INF ومجموعة من المتغيرات المستقلة الممثلة في المتغيرات SHEAL_POS, SHEAL_NEG, SUF_POS, SUF_NEG، وغيرها من المتغيرات ذات التأثير المحتمل على INF، لكل متغير تم تقديم قيمة المعامل (Coefficient)، الخطأ المعياري (Standard Error)، قيمة اختبار t (t-test)، وقيمة الاحتمال (P-value) لتقييم دلالة التأثير.
أولا، نلاحظ أن غالبية المتغيرات المستقلة تمتلك معاملات ذات قيم صغيرة نسبيا، مما يشير إلى تأثير محدود لكل منها على المتغير التابع INF على سبيل المثال، معامل SHEAL_POS هو 0.057 مع خطأ معياري 0.0566، واختبار t يساوي 1.007، مما يعطي قيمة احتمال 0.3600، وهذا يعني أن تأثير SHEAL_POS غير معنوي إحصائيا عند مستويات الثقة المعتادة (مثل 5% أو 10%).
نفس الأمر ينطبق على متغير SHEAL_NEG بمعامل -0.0189 وقيمة احتمال 0.4297، ومتشابه مع المتغيرات الأخرى SUF_POS وSUF_NEG وSUH_POS وSUH_NEG، التي تظهر كلها احتمالات مرتفعة تفوق 0.4، مما يدل على عدم وجود أدلة قوية تدعم وجود تأثير معنوي لهذه المتغيرات على INF.
مع ذلك، يبرز المتغير SUM_POS كمتغير قد يكون له تأثير ذي دلالة إحصائية جزئية، حيث أن معامل التأثير له قيمة سالبة (-0.195) مع اختبار t يساوي -2.206 وقيمة احتمال تساوي 0.0785، وهذه القيمة الاحتمالية تقع قريبة من مستوى الدلالة 0.10، مما يعني أنه يمكن اعتبار تأثير SUM_POS هامشيا أو محدود الأهمية عند مستوى ثقة 90%، هذا يشير إلى أن زيادة في SUM_POS قد ترتبط بانخفاض في INF، رغم أن العلاقة ليست قوية بدرجة كبيرة.
بقية المتغيرات مثل SUM_NEG وSUN_POS وSUN_NEG وSUP_POS وSUP_NEG تظهر جميعها معاملات ذات احتمالات مرتفعة تتجاوز 0.1، مما يعني عدم وجود تأثير إحصائي معنوي لها.
بالنسبة للثابت (C) فهو موجب وقيمته حوالي 10.3، مع خطأ معياري 5.99 وقيمة احتمال 0.1462، مما يشير إلى أن الثابت ليس معنويا عند مستويات الدلالة التقليدية.
ومنه يمكن القول إن نموذج NARDL هذا لا يظهر تأثيرات إحصائية قوية لمعظم المتغيرات المستقلة على المتغير التابع INF، ما يشير إلى عدم وجود علاقة سبب وتأثير ذات دلالة واضحة بين هذه المتغيرات في العينة المدروسة، المتغير الوحيد الذي يقترب من الدلالة الإحصائية هو SUM_POS، والذي قد يدل على تأثير سلبي محتمل على INF، يستحق دراسة أعمق أو تأكيدا من خلال بيانات إضافية
4-2-2-3: تشخيص صحة النموذج
يعتبر تشخيص صحة نموذج الانحدار الذاتي للإبطاء الموزع غير الخطي (NARDL) خطوة جوهرية في أي دراسة اقتصادية تطبيقية تهدف إلى تحليل العلاقات الديناميكية بين المتغيرات مع التمييز بين التأثيرات الإيجابية والسلبية فبعد تقدير نموذج NARDL، لا تقتصر أهمية القيم الإحصائية للمعاملات فقط، بل تتجاوز ذلك إلى التحقق من مدى استيفاء النموذج لافتراضاته الأساسية لضمان صلاحية النتائج وقوة الاستدلالات المستخلصة، ويشمل هذا التشخيص اختبارات متعددة مثل اختبار Breusch-Godfrey للكشف عن الارتباط الذاتي في أخطاء النموذج عبر الزمن، واختبار ARCH الذي يهدف إلى رصد وجود التباين غير المتساوي في الأخطاء، وهو أمر يؤثر على ثبات التقديرات بالإضافة إلى ذلك، يستخدم اختبار Ramsey RESET للتحقق من صحة الشكل الوظيفي للنموذج والتأكد من أن التمثيل الرياضي للعلاقات بين المتغيرات مناسب، بينما يعنى اختبار Jarque-Bera بفحص مدى توافق توزيع الأخطاء مع التوزيع الطبيعي، وهو شرط مهم لتطبيق الاختبارات الإحصائية التقليدية، إن إهمال استيفاء هذه الافتراضات قد يؤدي إلى تحيز في التقديرات ونتائج مضللة، مما يؤثر سلبا على اتخاذ القرارات الاقتصادية المبنية على تلك النتائج لذلك، فإن إجراء تشخيص شامل للنموذج NARDL يعزز من مصداقية الدراسة ويوفر قاعدة متينة لتفسير النتائج بطريقة علمية دقيقة وموثوقة
الجدول رقم7: اختبارات تشخيص وتأكد من صحة النموذج
تحليل نتائج اختبار نموذج NARDL يظهر مؤشرات مهمة حول صلاحية النموذج وفحص الفرضيات الأساسية المرتبطة به أولا، اختبار Breusch-Godfrey للكشف عن الارتباط التسلسلي (Serial Corrélation LM Test) أظهر قيمة F تساوي 2.2945 مع قيمة احتمالية (p-value) تبلغ 0.2485، مما يشير إلى عدم وجود دليل قوي على وجود ارتباط تسلسلي في البواقي عند مستوى دلالة 5%، وبالتالي يمكن اعتبار الفرضية الصفرية بعدم وجود ارتباط تسلسلي مقبولة.
ثانيا، اختبار Heteroskedasticity باستخدام ARCH يهدف إلى التحقق من وجود تغاير غير ثابت في البواقي (أي تغير تباين الخطأ عبر الزمن)، حيث بلغت قيمة F 1.0528 مع p-value قيمتها 0.3192، مما يشير إلى عدم وجود دليل على وجود مشكلة التغاير غير الثابت عند مستوى الثقة المعتاد، وبالتالي لا يرفض النموذج فرضية تجانس التباين.
ثالثا، اختبار Ramsey RESET الذي يفحص صحة التحديد البنيوي للنموذج، أظهر قيمة F تساوي 4.1085 مع p-value تصل إلى 0.1126، ما يعني أن النموذج لا يعاني من مشكلة التحديد الخاطئ (Misspecification) عند مستوى دلالة 5%، حيث أن قيمة الاحتمال أكبر من 0.05، وهذا يعزز الثقة في أن النموذج تم تحديده بشكل ملائم.
أخيرا، اختبار Jarque-Bera لفحص مدى تماثل وتوزيع البواقي يظهر قيمة إحصائية X² تبلغ 1.0808 مع p-value بقيمة 0.5825، مما يشير إلى أن البواقي تتبع توزيعا قريبا من التوزيع الطبيعي، حيث لا يمكن رفض الفرضية الصفرية بوجود توزيع طبيعي للبواقي.
بناء على هذه النتائج، يمكن القول إن نموذج NARDL الذي تم اختباره يتمتع بخصائص جيدة من حيث عدم وجود ارتباط تسلسلي، وعدم وجود تغاير في التباين، وعدم وجود أخطاء في تحديد النموذج، كما أن بواقي النموذج تتبع التوزيع الطبيعي المطلوب، مما يعزز من مصداقية النموذج وموثوقيته في تفسير العلاقة بين المتغيرات المدروسة
4-2-2-4: اختبار السكون للنموذج المقدر
يعد اختبار السكون خطوة أساسية ومهمة عند تطبيق نموذج الانحدار الذاتي للإبطاء الموزع غير المتناسق (NARDL)، حيث يشكل هذا الاختبار قاعدة ضرورية لفهم خصائص السلاسل الزمنية المستخدمة في التحليل، تكمن أهمية اختبار السكون في ضمان أن المتغيرات الداخلة في نموذج NARDL مستقرة إما عند المستوى أو بعد التفاضل، وهو أمر حيوي لتفادي مشكلة الانحدار الكاذب التي قد تؤدي إلى نتائج تقديرية مضللة، فالاستقرار أو السكون يعني أن خصائص السلسلة الزمنية مثل المتوسط والتباين لا تتغير مع مرور الزمن، وهو شرط جوهري لتحقيق استقرار النموذج وجودة التقدير، وبما أن نموذج NARDL قادر على التعامل مع متغيرات ذات درجات مختلفة من السكون (أي بعضها ساكن عند المستوى والبعض الآخر عند التفاضل)، فإن التأكد من خصائص السكون قبل تقدير النموذج يسمح بضبط عملية التحليل بشكل صحيح، ويضمن صحة اختبارات التكامل المشترك والارتباط طويل الأجل بين المتغيرات لذلك، يعتبر اختبار السكون المرحلة التمهيدية الحاسمة التي تؤسس لدقة النتائج وموثوقية الاستنتاجات الاقتصادية المستخلصة من نموذج NARDL

	 الشكل8: اختبار السكون للنموذج المقدر
تشير نتائج اختبار استقرار النموذج إلى أن كلا من الرسمين البيانيين للاحتمالات البينية المتعلقة بالمجموع التراكمي للبواقي المعاودة (CUSUM) والمجموع التراكمي لمربعات البواقي المعاودة (CUSUM of Squares) يقعان بالكامل داخل حدود فترات الثقة أو الإطار الحرج عند مستوى معنوية 5%، ويعد هذا مؤشرا قويا على استقرار معاملات النموذج المقدر، سواء على المدى القصير أو الطويل، إن بقاء هذين المنحنيين ضمن الحدود الحرجة يعني أن التغيرات في المعلمات عبر الزمن لا تتجاوز الحدود المقبولة إحصائيا، مما يعزز من موثوقية النموذج وبالتالي، فإن الاتساق الظاهر بين نتائج الأجل القصير والأجل الطويل يشير إلى وجود علاقة توازنية مستقرة بين المتغيرات المدروسة، ما يدعم صلاحية استخدام النموذج في التنبؤ والتحليل الاقتصادي
5-1: النتائج المقترحة: 
5-1-1:  أهمية نموذج الانحدار الذاتي غير الخطي للإبطاء الموزع (NARDL) في تحليل العلاقات الاقتصادية قصيرة الأجل:
يعد نموذج NARDL أداة حديثة وفعالة تتيح دراسة العلاقة غير المتناظرة بين المتغيرات الاقتصادية، حيث يمكنه تمييز التأثيرات المختلفة للزيادات والانخفاضات في المتغيرات المستقلة على المتغير التابع، مع الاستناد إلى وجود توازن طويل الأجل مثبت مسبقا عبر اختبار حدود التكامل المشترك.
5-1-2  آلية تصحيح الخطأ (ECM) ودورها في الديناميكيات قصيرة الأجل:
يشتمل نموذج NARDL على آلية تصحيح الخطأ التي تعكس سرعة استجابة المتغيرات لأي انحراف عن العلاقة التوازنية طويلة الأجل، مما يتيح فهما دقيقا للتغيرات المؤقتة والديناميكيات الاقتصادية قصيرة الأجل، لا سيما في بيئات تتميز بعدم اليقين والتقلبات الاقتصادية المتكررة.
5-1-3 تباين الأثر بين المتغيرات في النموذج:
تظهر نتائج التقدير أن بعض المتغيرات مثل التضخم المتأخر (INF(-1)*) وSUM_POS وSUN_POS تمتلك تأثيرات معنوية إحصائيا على المتغير التابع، في حين أن غالبية المتغيرات الأخرى لا تظهر تأثيرات معنوية خلال الفترة المدروسة، مما يشير إلى أهمية تركيز التحليل والسياسات على تلك المتغيرات ذات التأثير الحقيقي.
5-1-4  التأثير السلبي للتضخم المتأخر على المتغير التابع:
يتضح من نتائج النموذج أن ارتفاع التضخم في الفترة السابقة يميل إلى خفض المتغير التابع في الفترة الحالية، وهو تأثير ذو دلالة إحصائية هامشي أو شبه معنوي عند مستوى الثقة المعتاد، ما يستوجب اعتباره في الدراسات والتحليلات المستقبلية.
5-1-5: استقرار النموذج وصلاحيته الإحصائية:
تشير اختبارات تشخيص صحة النموذج مثل اختبار الارتباط الذاتي للبواقي (Breusch-Godfrey)، اختبار التغاير غير الثابت (ARCH)، اختبار التحديد البنيوي (Ramsey RESET)، واختبار التوزيع الطبيعي للبواقي (Jarque-Bera) إلى أن النموذج مستوف لافتراضاته الأساسية، مما يعزز موثوقية النتائج وقوة الاستدلالات الاقتصادية المستخلصة منه.
5-1-6: استقرار معاملات النموذج على المدى القصير والطويل:
تؤكد نتائج اختبار الاستقرار باستخدام اختبارات CUSUM وCUSUM of Squares أن معاملات النموذج تبقى ضمن الحدود الحرجة لفترات الثقة، مما يدل على استقرار النموذج ديناميكيا، ويؤكد وجود علاقة توازنية مستقرة بين المتغيرات على المدى القصير والطويل، ويدعم استخدام النموذج في التنبؤات والتحليلات الاقتصادية.
5-1-7: التركيز على المتغيرات ذات الأثر الإحصائي في صياغة السياسات:
يبرز من نتائج النموذج أن التأثيرات ذات الدلالة الإحصائية تتركز على عدد محدود من المتغيرات، وهو ما يشير إلى ضرورة توجيه جهود التحليل وصنع القرار الاقتصادي نحو فهم وتطوير السياسات المرتبطة بهذه المتغيرات لتحقيق استجابة أكثر فعالية في المدى القصير والطويل.
5-2: المقترحات:
-ضبط التحويلات الاجتماعية وفقًا للظروف الاقتصادية لتفادي تأثيرها التضخمي خاصة في فترات الزيادة المفرطة؛
- ربط الزيادات في التحويلات بقدرة الاقتصاد الإنتاجية لضمان استدامتها دون تغذية التضخم؛
- اعتماد آلية استهداف دقيق للتحويلات (مثل الدعم الموجه) لتقليل أثرها الفوري على الأسعار؛
- تعزيز الرقابة على الأسواق خلال فترات رفع التحويلات لمنع المضاربة وارتفاع الأسعار؛
تحويل جزء من التحويلات الاجتماعية نحو الاستثمار في رأس المال البشري (الصحة، التعليم، التكوين المهني) لتعزيز النمو وتقليل الآثار التضخمية مستقبلاً؛
إعادة هيكلة النظام الاجتماعي ليكون أكثر كفاءة وأقل تكلفة، دون المساس بالحماية الاجتماعية الأساسية.
دراسة الأثر غير المتماثل عبر نماذج اقتصادية متقدمة (مثل نموذج NARDL)
تحليل الفترات السياسية والاقتصادية الحساسة التي شهدت تغيرات حادة في التحويل
6-2 - افاق الدراسة 
في ضوء النتائج التي توصلت إليها هذه الدراسة حول أثر التحويلات الاجتماعية على ديناميكية معدلات التضخم في الجزائر خلال الفترة 2000-2022 باستخدام نموذج NARDL، تبرز عدة آفاق بحثية مستقبلية يمكن أن تسهم في تعميق الفهم حول العلاقة بين السياسات الاجتماعية والاستقرار الاقتصادي. من بين هذه الآفاق، يمكن التوسع في إدراج متغيرات اقتصادية واجتماعية إضافية مثل البطالة، الفقر، والنمو الاقتصادي، لدراسة الأثر المشترك على التضخم. كما يُمكن تحليل التحويلات الاجتماعية بشكل أكثر تفصيلًا من خلال تصنيفها حسب نوعها (إعانات بطالة، دعم سكني، دعم غذائي...) لقياس تأثير كل فئة على حدة. ومن جهة أخرى، قد تكون الدراسات المقارنة مع دول مغاربية أو نامية ذات سياسات اجتماعية مماثلة مفيدة لفهم خصوصية الحالة الجزائرية. وبالإضافة إلى ذلك، يُستحسن دراسة دور التحويلات الاجتماعية خلال الفترات الاستثنائية مثل الأزمات الصحية أو الاقتصادية، لا سيما أزمة كوفيد-19 وتقلبات أسعار النفط، لمعرفة مدى تأثيرها في احتواء أو تغذية الضغوط التضخمية. كما يمكن الاستفادة من نماذج اقتصادية أكثر تطورًا، على غرار نماذج VAR غير الخطية أو تقنيات الذكاء الاصطناعي، لتعزيز دقة التحليل والتنبؤ. وأخيرًا، يُعد الفصل بين التضخم الكلي والتضخم الأساسي خطوة مهمة في الدراسات المستقبلية لفهم أثر التحويلات الاجتماعية على المكونات الهيكلية للتضخم بشكل أدق.
الملحق رقم(1) اختبار معيار اكايكي AIC 
الملحق رقم(2) نتائج اختبار الحدود 
الملحق رقم(3) تراجع تصحيح الخطأ المشروط 
الملحق رقم(4) جدول نتائج تقدير نموذج انحدار في المستويات 
الملحق رقم (5) مدرج تكراري لبواقي السلسلة الزمنية 
الملحق رقم (6) اختبار بريوش-جودفري للارتباط التسلسلي 
الملحق رقم (7) اختبار تجانس التباين: نموذج
الملحق رقم (8) اختبار تجانس التباين: نموذج 
الملحق رقم (9): منحنى المجموع التراكمي   
الملحق رقم (10): منحنى المجموع التراكمي للمربعات 
الملحق(11) اختبار عدم التماثل 
خلاصة
توصلت هذه الدراسة إلى أن نموذج الانحدار الذاتي غير الخطي للإبطاء الموزع يُعد أداة فعالة في تحليل العلاقات الاقتصادية غير المتماثلة، حيث مكّن من إبراز التأثيرات المختلفة للتحويلات الاجتماعية على معدلات التضخم في الجزائر خلال الفترة 2000-2022. وأظهرت نتائج التقدير وجود تأثيرات معنوية لبعض المتغيرات مثل التضخم المتأخر والتحويلات الإيجابية، ما يدل على دورها الحاسم في تفسير سلوك التضخم على المدى القصير. كما بينت الدراسة أن التضخم المتأخر يمارس تأثيرًا سلبيًا على المتغير التابع، ما يعكس طبيعة الاستجابة غير المتناظرة للاقتصاد تجاه التغيرات السعرية. وأكدت اختبارات الاستقرار والتشخيص الإحصائي سلامة النموذج وقوة نتائجه التفسيرية. ومن الناحية التطبيقية، توضح الدراسة أن السياسات الاجتماعية ينبغي أن تأخذ بعين الاعتبار هذه العلاقة غير المتماثلة من خلال ضبط التحويلات وفقًا للظروف الاقتصادية وتوجيهها نحو القطاعات المنتجة، بما يضمن استدامتها دون تغذية التضخم. كما دعت إلى تعزيز آليات الاستهداف والدقة في توزيع الدعم، مع ضرورة دراسة الفترات الاقتصادية الحساسة ومواصلة التحليل باستخدام نماذج أكثر تطورًا مستقبلاً. وعليه، فإن هذه الدراسة تسهم في تقديم إطار تحليلي مفيد لصانعي القرار من أجل تحقيق توازن فعّال بين العدالة الاجتماعية والاستقرار الاقتصادي في الجزائر
This study found that the Nonlinear Autoregressive Distributed Lag (NARDL) model is an effective tool for analyzing asymmetric economic relationships, allowing for the identification of differing effects of social transfers on inflation rates in Algeria over the period 2000–2022. The estimation results revealed that certain variables, such as lagged inflation and positive shocks to social transfers, have statistically significant impacts on the dependent variable, highlighting their critical role in explaining short-term inflation dynamics. Moreover, the negative effect of past inflation on current outcomes indicates an asymmetric response of the economy to price fluctuations. The model also passed key diagnostic and stability tests, confirming its robustness and reliability. From a policy perspective, the findings suggest that social transfers should be adjusted according to economic conditions and redirected toward productive sectors to ensure their sustainability without fueling inflation. The study also emphasizes the importance of targeted support mechanisms, especially during sensitive political and economic periods, and encourages the use of more advanced models in future research. Overall, this study provides a useful analytical framework for policymakers aiming to strike a balance between social equity and macroeconomic stability in Algeria.

\end{document}
